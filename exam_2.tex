\input{boilerplate.tex}

\date{}

\begin{document}

\Begin{defn}
\textbf{Кольцом} называется непустое множество \(K\) с операциями сложения и умножения, обладающими следующими свойствами:

\begin{itemize}
\tightlist
\item
  относительно сложения \(K\) есть абелева группа (называемая \textbf{аддитивной группой} кольца \(K\));
\item
  \(a(b+c)\) = \(ab+ac\) и \((a+b)c = ac+bc\) для любых \(a,b,c \in K\) (\emph{дистрибутивность умножения относительно сложения}).
\end{itemize}

Кольцо \(K\) называется \textbf{ассоциативным}, если умножение в нем ассоциативно, т. е. \((ab)c = a(bc)\) для любых \(a, b, c \in K\).

Кольцо \(K\) называется \textbf{кольцом с единицей}, если в \(K\) существует нейтральный элемент относительно умножения, обозначаемый обычно через \(1\), т.е. \(1a=a1=a\) для любого \(a\in K\).

Кольцо \(K\) называется \textbf{коммутативным}, если \(K\) --- ассоциативное кольцо с единицей, в котором умножение коммутативно, т. е. \(ab = ba\) для любых \(a, b \in K\).

\textbf{Полем} называется коммутативное кольцо, содержащее не менее двух элементов, в котором всякий ненулевой элемент обратим.
\End{defn}

\Begin{defn}
Элемент \(a^{-1}\) кольца с единицей называется \textbf{обратным} к элементу \(a\), если \(aa^{-1} = a^{-1}a = 1\). Элемент, имеющий обратный, называется \textbf{обратимым} или \textbf{единицей кольца}. Множество \(K^*\) обратимых элементов кольца \(K\) является группой по умножению. Она называется \textbf{мультипликативной группой} или \textbf{группой обратимых элементов} кольца \(K\).
\End{defn}

\Begin{defn}
Элемент \(a\neq 0\in K\) называется \textbf{делителем нуля}, если найдется такой элемент \(b \neq 0\), что \(ab=0\).
\End{defn}

\Begin{defn}
\textbf{Гомоморфизмом} коммутативных колец называется отображение \(\varphi: K \to L\), при котором сохраняются операции, то есть для любых \(a,b \in K\) выполнены равенства \(\varphi(a+b)=\varphi(a)+\varphi(b)\), \(\varphi(a\cdot b)=\varphi(a)\cdot \varphi(b)\), \(\varphi(1)=1\). Аналогично определяется \textbf{изоморфизм} колец (это гомоморфизм + биекция).
\End{defn}

\Begin{defn}
Коммутативное кольцо без делителей нуля называется \textbf{областью целостности} или \textbf{кольцом целостности}.
\End{defn}

\Begin{defn}
Пусть \(K\) --- область целостности. Будем говорить, что элемент \(a\in K\) \textbf{делит} элемент \(b\in K\), если найдётся такое \(r \in K\), что \(ar=b\).

Группа обратимых элементов \(K^*\) действует на всё кольцо \(K\) умножениями слева. Элементы, находящиеся в одной орбите этого действия, будем называть \textbf{ассоциированными}, а сами орбиты --- \textbf{классами ассоциированности}.

То есть элементы \(x\) и \(y\) кольца \(K\) являются \textbf{ассоциированными}, если найдётся такое \(r \in K^*\), что \(x=ry\). Обозначение: \(x \sim y\).
\End{defn}

\Begin{defn}
Пусть K --- область целостности. Необратимый ненулевой элемент \(x \in K\) называется \textbf{неразложимым}, если из равенства \(x=ab\) следует, что либо \(a \in K^*\), либо \(b \in K^*\).
\End{defn}

\Begin{defn}
Пусть K --- область целостности. Назовём ненулевой необратимый элемент \(x \in K\) \textbf{простым}, если из того, что \(x \mid ab\), следует, что либо \(x \mid a\), либо \(x \mid b\).
\End{defn}

\Begin{defn}
Кольцо \(K\) называется \textbf{евклидовым}, если существует такое отображение (\textbf{норма}) \(N : K \setminus \{0\} \rightarrow \Z_{\ge 0}\), что для любых \(a,b \in K \setminus \{0\}\) выполнены два условия:
1. \(N(ab) \ge N(a)\);
2. найдутся такие элементы \(q , r\in K\), что \(a=qb+r\) и либо \(r=0\), либо \(N(r) < N(b)\).

\End{defn}

\Begin{defn}
Область целостности \(K\) называется \textbf{факториальным кольцом}, если выполнены следующие два условия:
1. (\textbf{существование разложения}) любой элемент \(x \in K\), \(x \neq 0\) представляется в виде произведения неразложимых элементов с точностью до ассоциированности, то есть \(x = u p_1 \ldots p_k\), где \(u \in K^*\), \(p_i\) --- неразложимые элементы;
2. (\textbf{единственность разложения}) данное разложение единственно в следующем смысле. Пусть \textbackslash{}\(x = u p_1 \ldots p_k = w q_1 \ldots q_l\) --- два разложения, где \(u,w\) обратимы, а \(p_i,q_j\) --- неразложимые элементы. Тогда \(k=l\) и элементы \(q_j\) можно перенумеровать так, чтобы для всех \(i\) элементы \(p_i\) и \(q_i\) были ассоциированны.

\End{defn}

\Begin{defn}
\textbf{Корнем из единицы степени n} называется \(z \in \C: z^n=1\).

Корень из 1 степени 3, находящийся в верхней полуплоскости, обозначается \(\omega\).

\(\forall u \in \C\) определим \(\Z[u] = \bigcup\limits_{n=0}^\infty \{a_o+a_1u+\dots+a_nu^n \mid a_o, \dots, a_n \in \Z \}\) --- \textbf{множество, порождённое элементом \(u\) над \(\Z\)}.

Тогда \(\Z[\omega]\) --- \textbf{числа Эйзенштейна}.
\End{defn}

\Begin{defn}
\textbf{Наибольший общий делитель (НОД)} чисел \((a, b)\) двух элементов \(a, b \in K\) --- области целостности, есть их общий делитель, который делится на все их другие общие делители.
\End{defn}

\Begin{defn}
\textbf{Подкольцо} \(S \subset K\) есть подгруппа по сложению, замкнутая относительно умножения (т. е. \(\forall a, b \in S \ ab \in S\)).
\End{defn}

\Begin{defn}
\textbf{Идеал} \(I\) коммутативного кольца \(K\) --- это такое множество элементов, что

\begin{enumerate}
\def\labelenumi{\arabic{enumi}.}
\tightlist
\item
  \((I,+) \subset (K,+)\) --- подгруппа по сложению.
\item
  Для любых элементов \(a\in K\) и \(x \in I\) верно, что \(ax \in I\).
\end{enumerate}

Таким образом, подкольцо \(I \subset K\) называется идеалом, если \(\forall a\in K, x \in I \Have xy \in I\).
\End{defn}

\Begin{defn}
\(0 \subset K, K \subset K\) --- идеалы. Они называются \textbf{тривиальными}.
\End{defn}

\Begin{defn}
\((a_1,\ldots,a_n) = \{a_1x_1+\ldots+a_nx_n \mid x_1,\ldots,x_n \in K\}\) --- \textbf{идеал, порождённый элементами \(a_1,\ldots,a_n\)}.
\End{defn}

\Begin{defn}
\textbf{Конечно порождённый идеал} --- идеал, порождённый конечным количеством элементов.
\End{defn}

\Begin{defn}
\((a) = \{ax \mid x\in K\}\) --- \textbf{главный идеал} или \textbf{идеал, порождённый одним элементом}.
\End{defn}

\Begin{defn}
Область целостности, в которой все идеалы главные, называется \textbf{кольцом главных идеалов} (сокращённо КГИ).
\End{defn}

\Begin{defn}
Назовём нетривиальный идеал \(I\) \textbf{простым}, если \(ab \in I \Then \begin{sqcases} a\in I \\ b \in I \end{sqcases}\).
\End{defn}

\Begin{defn}
Назовём нетривиальный идеал \(I\) \textbf{максимальным}, если он максимальный по включению, то есть не существует идеала \(J\) такого, что \(I \subsetneq J \subsetneq K\).
\End{defn}

\Begin{defn}
Будем называть многочлен \(f \in K[x]\) \textbf{примитивным}, если его коэффициенты взаимно просты.
\End{defn}

\Begin{defn}
\textbf{Расширением полей} называется вложение полей \(K \supset F\).

\textbf{Вложение} --- инъективное отображение \(F \to K\).
\End{defn}

\Begin{defn}
Элемент \(\alpha \in K\) \textbf{алгебраичен} над \(F\), если выполнено одно из двух эквивалентных условий:

\begin{enumerate}
\def\labelenumi{\arabic{enumi}.}
\tightlist
\item
  расширение \(F(\alpha) \supset F\) конечно;
\item
  \(\alpha\) --- корень многочлена \(f(x) \in F[x]\).
\end{enumerate}

\End{defn}

\Begin{defn}
\textbf{Трансцендентный элемент} --- элемент, не являющийся алгебраическим.
\End{defn}

\Begin{defn}
Расширение \(K \supset F\) называется \textbf{алгебраическим}, если оно состоит из элементов, алгебраических над \(F\).
\End{defn}

\Begin{defn}
\textbf{Пример алгебраического расширения поля.}

\begin{itemize}
\tightlist
\item
  Расширение \(\C \supset \R\) является алгебраическим.
\item
  Расширение \(F \supset F\) является алгебраическим.
\item
  Любое конечное расширение \(K \supset F\) является алгебраическим.
\end{itemize}

\textbf{Примеры алгебраических элементов (не нужно в вопросе, на всякий случай).}

\begin{itemize}
\tightlist
\item
  В расширении \(\Q \subset \C\) элементы \(\sqrt{2}, \sqrt{3}, \sqrt[n]{7}, i, i+\sqrt{3}\) поля \(\C\) --- алгебраические над \(\Q\).
\item
  В \(\R \subset \C\) все элементы алгебраичны над \(\R\).
\item
  В любом расширении \(K \supset F\) элементы \(F\) являются алгебраическими над \(F\).
\end{itemize}

\End{defn}

\Begin{defn}
\textbf{Пример не алгебраического расширения поля.}

Если в расширении есть трансцендентный элемент, оно не алгебраическое.

\begin{itemize}
\tightlist
\item
  В расширении \(\Q \subset \C\) элементы \(\pi, e\) трансцендентны над \(\Q\).
\item
  В расширении \(F \subset F(x)\) элемент \(x\) --- трансцендентный над \(F\). Тут \(x\) обязательно должен быть \(x\) многочленовым \(x\), а не элементом \(F\) (контрпример: рассмотрим расширение \(\Q(\sqrt{2}) \supset \Q\), тогда многочлен \(x^2-2\) имеет своим корнем \(\sqrt{2}\) \(\Then\) \(\sqrt{2}\) алгебраический). {[}непонятно, ждём комментария Ильинского{]}
\end{itemize}

\End{defn}

\Begin{defn}
Для данного элемента \(\alpha \in K\) назовём \textbf{минимальным многочленом} многочлен \(m_{\alpha} = m_{\alpha,F}\) со старшим коэффициентом \(1\), удовлетворяющий одному из СЭУ (следующих эквивалентных условий):

\begin{enumerate}
\def\labelenumi{\arabic{enumi}.}
\tightlist
\item
  Для идеала \(I_{\alpha, F} := (m_{\alpha, F})\) выполнено \(I_{\alpha,F} = \{f(x) \in F[x] \mid f(\alpha)=0\}\);
\item
  \(m_{\alpha}\) --- многочлен из \(I_{\alpha}\) минимальной степени;
\item
  \(m_{\alpha}\) --- неприводимый многочлен из \(I_{\alpha}\).
\end{enumerate}

\End{defn}

\Begin{defn}
Через \(F(\alpha_1,\ldots,\alpha_n)\) обозначим минимальное поле в \(K\), содержащее \(F\) и \(\alpha_1,\ldots,\alpha_n\).
\End{defn}

\Begin{defn}
Назовём \textbf{полем разложения} многочлена \(f(x)\) над полем \(F\) такое расширение \(L \supset F\), что \(L\) содержит все корни многочлена \(f(x)\) и не существует нетривиального подполя \(K \subset L\), удовлетворяющего тому же условию.
\End{defn}

\Begin{defn}
Поле \(K\) называется \textbf{алгебраически замкнутым}, если выполнено одно из следующих эквивалентных условий:

\begin{enumerate}
\def\labelenumi{(\arabic{enumi})}
\tightlist
\item
  Любое алгебраическое расширение над \(K\) тривиально.
\item
  Любой многочлен \(f(x) \in K[x]\) с \(\deg f(x) \geq 1\) имеет корень в \(K\).
\item
  Любой многочлен \(f(x) \in K[x]\) с \(\deg f(x) \geq 1\) раскладывается на линейные множители в \(K\).
\item
  Все неприводимые над \(K\) многочлены имеют степень \(1\).
\item
  Для любого многочлена \(f(x) \in K[x]\) с \(\deg f(x) \geq 1\) его поле разложения совпадает с \(K\).
\end{enumerate}

\End{defn}

\Begin{defn}
\textbf{Алгебраическим замыканием} поля \(F\) называется алгебраическое расширение \(K=\overline{F}\) поля \(F\), которое является алгебраически замкнутым полем.

\End{defn}

\Begin{defn}
Даны точки \(0\) и \(1\) комплексной плоскости. Точку \(x \in \C\) \textit{можно построить}, если найдётся такая последовательность точек \(x_0=0\), \(x_1=1\), \(\ldots\), \(x_n=x\), где точка \(x_k\) получается из точек \(\{x_0,x_1,\ldots, x_{k-1}\}\) при помощи применения трёх следующих действий:

\begin{enumerate}
\def\labelenumi{\arabic{enumi}.}
\tightlist
\item
  Провести прямую через ранее построенные точки.
\item
  Провести окружность с центром в уже построенной точке, проходящую через другую построенную точку.
\item
  Построить точку пересечения двух \emph{различных} прямых, прямой и окружности, двух \emph{различных} окружностей, полученных в результате действий 1 и 2.
\end{enumerate}

\End{defn}

\Begin{defn}
Обозначим через \(\xi_n\) \textbf{примитивный корень} \(n\)-ой степени из \(1\), то есть корень многочлена \(x^n-1\), который не является корнем многочлена \(x^k-1\) при \(k<n\).
\End{defn}

\Begin{defn}
Алгебраические над \(F\) элементы \(\alpha\) и \(\beta\) называются \textbf{сопряженными}, если \(m_{\alpha,F} = m_{\beta,F}\) или \(m_{\alpha,F}(\beta)=0\).
\End{defn}

\Begin{defn}
\textbf{Признак неприводимости Эйзенштейна}

Пусть \(f(x)\) --- многочлен с целыми коэффициентами и существует такое простое число \(p\), что:

\begin{enumerate}
\def\labelenumi{\arabic{enumi}.}
\tightlist
\item
  старший коэффициент \(f(x)\) не делится на \(p\);
\item
  все остальные коэффициенты \(f(x)\) делятся на \(p\);
\item
  свободный член \(f(x)\) не делится на \(p^2\).
\end{enumerate}

Тогда многочлен \(f(x)\) неприводим над полем рациональныx чисел.

\textbf{Более общая формулировка из Lecture\_all.pdf:}

Пусть \(F\) --- факториальное кольцо, \(I \subset F\) --- простой идеал, \(f(x) = a_nx^n+\dots+a_1x + a_0 \in F [x]\) --- многочлен степени \(n > 1\).
Если \(a_0, a_1, \dots, a_{n-1} \in I, a0 \not\in I^2, a_n \not\in I\), то у \(f(x)\) нет делителей степени \(d\) при \(1 \le d\le n-1\).

\End{defn}

\Begin{defn}
Пусть \(F \subset K\) --- расширение полей. Множество автоморфизмов \(K\), оставляющих \(F\) на месте является группой и называется \textbf{группой автоморфизмов} и обозначается \(\Aut_F(K) = \Aut([K:F])\). Если \(F\) --- основное поле (\(\Q\) или \(\Z_p\)), то символ \(F\) опускают.
\End{defn}

\Begin{defn}
Пусть \(H \subset \Aut_F(K)\) --- подгруппа. Тогда \(K^{H} = \{x \in K \mid \forall h \in H \, h(x)=x\}\) является полем, причём \(K \supset K^H \supset F\).
\End{defn}

\Begin{defn}
Пусть \(K \supset F\) --- конечное расширение. Будем называть это расширение \textbf{нормальным}, или \textbf{расширением Галуа} если выполнено одно из следующих эквивалентных условий:

\begin{enumerate}
\def\labelenumi{(\arabic{enumi})}
\tightlist
\item
  Вместе с каждым элементом поле \(K\) содержит и все сопряженные;
\item
  \(K\) --- поле разложение многочлена \(f(x) \in F[x]\);
\item
  \(|\Aut_F{K}| = [K:F]\);
\item
  \(K^{\Aut_F{K}}=F\).
\end{enumerate}

\End{defn}

\Begin{defn}
Группа автоморфизмов расширения Галуа \(K\), сохраняющих \(F\), \(\Aut_FK\), называется \textbf{группой Галуа} \(\Gal_FK\).
\End{defn}

\end{document}
