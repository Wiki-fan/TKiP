\input{boilerplate.tex}

\date{}

\begin{document}

\begin{problem}[1 (1.1)]
Для любых $a,b,c \in K$ выполнены равенства
\end{problem}

\begin{enumerate}
\def\labelenumi{\alph{enumi})}
\i \(a0=0a=0\)
\i \(a(-b)=(-a)b=-ab\)
\i \((a-b)c = ac-bc\) и \(a(b-c)=ab-ac\)
\end{enumerate}

\begin{solution}
\begin{enumerate}
\def\labelenumi{\alph{enumi})}
\i
  \(a0=a(0+0)=a0+a0 \Then a0=0\)

  \(0a=0\) --- аналогично.
\i
  \(0=a0=a(b+(-b))=ab+a(-b) \Then -ab = a(-b)\)
\i
  \((a-b)c+bc = (a-b+b)c=ac \Then (a-b)c = ac-bc\)

  \(a(b-c)+ac=a(b-c+c)=ab \Then a(b-c)=ab-ac\)
\end{enumerate}
\end{solution}

\begin{problem}[2(1.2)]
\end{problem}

\begin{enumerate}
\def\labelenumi{\alph{enumi})}
\i
  В кольце не может быть двух различных единиц.
  \begin{solution}
  \(1_1 \ubrace{=}{\text{т. к.} 1_2 \text{--- единица}} 1_1 \cdot 1_2 \ubrace{=}{\text{т. к.} 1_1 \text{--- единица}} 1_2\)
  \end{solution}
\i
  Пусть кольцо с единицей содержит не меньше двух элементов. Тогда \(1 \neq 0\).
  \begin{solution}
  \(\forall e \ne a \in K\ a \ubrace{=}{\text{св-во единицы}} a \cdot e \ubrace{=}{\text{св-во нуля}} e\)
  \end{solution}
\i
  Может ли элемент ассоциативного кольца иметь более одного обратного элемента?
  \begin{solution}
  Пусть \(a_1 \ne a_2\) --- обратные к \(a\) элементы. Тогда
  \(a_1 a a_2 = \begin{cases} a_1 \cdot 1 = a_1 \\ 1 \cdot a_2 = a_2 \end{cases}\)

  Получается, они равны.
  \end{solution}
\end{enumerate}

\begin{problem}[3(1.3, 2.4)]
Уметь отвечать на вопросы: является ли данное кольцо K коммутативным? ассоциативным? кольцом с единицей? область целостности? поле? евклидово кольцо? Какие в K есть обратимые элементы? неразложимые? простые?
\end{problem}

\begin{problem}[4 (2.1(в))]
Обратимый элемент кольца не может быть делителем нуля.
\end{problem}

\begin{solution}
Пусть \(a \in K\) обратим, \(\exists a^{-1} \in K: aa^{-1} = 1\). Если \(a\) --- делитель нуля, то \(\exists 0 \ne b \in K: ab=0\). Тогда \(a^{-1} a b = \begin{cases} a^{-1} \cdot 0 = 0 \\ 1 \cdot b = b \ne 0 \end{cases}\).
Противоречие.
\end{solution}

\begin{problem}[5(2.1(д))]
Если $K$ --- кольцо без делителей нуля, то возможно сокращение: если $ac=bc$ и $c \neq 0$, то $a=b$.
\end{problem}

\begin{solution}
\(ac=bc \Iff (a-b)c=0 \Then\) т. к. нет делителей нуля и \(c \ne 0\), д. б. \(a-b=0\), т. е. \(a=b\).
\end{solution}

\begin{problem}[6(2.1(г))]
В конечном коммутативном кольце если ненулевой элемент не является делителем нуля, то он обратим.
\end{problem}

\begin{solution}
Кольцо конечно \(\Then\) его элементы можно занумеровать: \(a_1, \dots, a_n\). $\forall a \ne 0$ элементы \(a\cdot a_1, \dots, a \cdot a_n\) должны быть все разные (иначе \(\exists i \ne j: \ a\cdot a_i = a \cdot a_j \Then \ubrace{a}{\ne 0} \ubrace{(a_i-a_j)}{\ne 0\text{, т. к. }i \ne j} = 0\), т. е. \(a\) --- делитель нуля).

Тогда \(\exists i: a \cdot a_i = 1\), т. к. \(1 \in K\) (т. е. \(a\cdot a_1, \dots, a \cdot a_n\) --- \(n\) разных элементов кольца, а в кольце всего \(n\) элементов; значит, какое-то \(aa_i\) должно быть \(1\)).
\end{solution}

\begin{problem}[7]
Конечная область целостности (состоящая из более чем одного элемента) --- поле.
\end{problem}

\begin{solution}
В области целостности нет делителей нуля, а если в конечном коммутативном кольце элемент --- не делитель нуля, то он обратим (№6). Т. е. все элементы обратимы.

Имеем \(\ge 2\) элементов по условию.
\end{solution}

\begin{problem}[8]
Множество $K^*$ обратимых элементов коммутативного кольца K является группой по умножению. Она называется \bf{мультипликативной группой}, или \bf{группой обратимых элементов} кольца $K$.
\end{problem}

\begin{solution}
Пусть \(K\) --- кольцо, \(a, b \in K^*\). Тогда \(\exists a^{-1}, b^{-1} \in K^*\). Проверим групповые свойства.

\begin{enumerate}
\def\labelenumi{\arabic{enumi}.}
\tightlist
\i
  a(bc) = (ab)c --- ассоциативность в \(K^*\) следует из свойств кольца \(K\).
\i
  \(\exists 1 \in K^*\) (единица из $K$ будет единицей из $K^*$: она лежит в $K^*$, т. к. $\exists 1^{-1} = 1$, ибо $1\cdot 1 = 1$, и выполняется свойство единицы $a1=1a=a$).
\i
  \((b^{-1}a^{-1})(ab) = (ab)(b^{-1}a^{-1})=1 \Then (ab)^{-1} = b^{-1} a^{-1} \in K^*\) --- обратимость.
\i
  \(\forall a, b \in K \Have ab \in K^*,\) т. к. $\exists (ab)^{-1} = b^{-1}a^{-1}$.
\end{enumerate}

Значит, \(K^*\) --- группа по умножению.
\end{solution}

\begin{problem}[9(1.5-1.7)]
Базовые знания про комплексные числа: сложение, умножение, модуль, аргумент, извлечение корней n-ой степени.
\end{problem}

\begin{solution}
\bf{Компл'ексное число} \(z\) --- это выражение вида \(z=a+bi\), где \(a\) и \(b\) --- числа из \(\R\), а \(i\) --- \bf{мнимая единица}. По определению \(i^2=-1\).
Число \(a\) называют \bf{вещественной частью} комплексного числа \(z\) (пишется \(a={\rm Re}\,(z)\)), а число \(b\) --- \bf{мнимой частью} \(z\) (пишется \(b={\rm Im}\,(z)\)).
Комплексные числа можно складывать и умножать, \lgq раскрывая скобки и приводя подобные\rgq. Множество комплексных чисел обозначают буквой \(\C\).

Каждому комплексному числу \(z=a+bi\) сопоставим точку \((a,b)\) и вектор \((a,b)\). Длина этого вектора называется \bf{модулем} числа \(z\) и обозначается \(|z|\). Пусть \(z\ne0\).
Угол (в радианах), отсчитанный против часовой стрелки от вектора \((1,0)\) до вектора \((a,b)\), называется \bf{аргументом} числа \(z\) и обозначается \({\rm Arg}\,(z)\). Аргумент определен с точностью до прибавления числа вида \(2\pi n\), где \(n\in\Z\).

\bf{Тригонометрическая форма записи.} Для любого ненулевого комплексного числа \(z\) имеет место равенство\\ \(z=r (\cos \varphi +i\sin \varphi )\), где \(r=|z|\), \(\varphi={\rm Arg}\,(z)\).

Для комплексного числа \(z=r(\cos\varphi+i\sin\varphi)\) и натурального числа \(n\in\N\) выполнена \bf{формула Муавра}\\
\(z^n=r^n(\cos n\varphi+i\sin n\varphi)\).

Для комплексного числа \(z=a+bi\), где \(a,b\in \R\) число \(\overline{z}=a-bi\) называется \bf{комплексно-сопряжённым} к \(z\). Выполнены следующие равенства:

\(|z|^2=z\overline{z}\), \(\overline{z+w} = \overline{z} + \overline{w}\), \(\overline{zw}= \overline{z}\overline{w}\).

\end{solution}

\begin{problem}[10(2.2)]
\end{problem}

\begin{enumerate}
\def\labelenumi{\alph{enumi})}
\i
  Докажите, что для элементов $x, y$ области целостности $K$ следующие условия равносильны:

  \begin{enumerate}
  \def\labelenumii{(\arabic{enumii})}
  \tightlist
  \i
    \(x \sim y\);
  \i
    \(x \mid y\) и \(y \mid x\);
  \i
    множество делителей \(x\) и множество делителей \(y\) равны.
  \end{enumerate}

  \begin{solution}

  \begin{itemize}
  \tightlist
  \i
    \((1) \Then (2):\)
    \(\exists r \in K^*: x=ry \Then y|x\) по определению.
    Т. к. \(r \in K^*\), \(\exists r^{-1} \in K^*: r^{-1}x=y \Then x|y\) по определению.
  \i
    \((2) \Then (3):\)
    Пусть \(x|y, x \Divby a\), т. е. $a$ --- делитель $x$. Тогда \(\begin{cases} \exists c: y = xc\\ \exists b: x = ab \end{cases}\) (по опр.) \(\Then\) \(y = xc = abc = a(bc) \Then y \Divby a\).
  \i
    \((3) \Then (2):\)
    Множества делителей \(x\) и \(y\) совпадают, \(x|x \Then x\) будет во множестве делителей \(y\), т. е. \(x|y\). Симметрично, \(y|x\).
  \i
    \((2) \Then (1):\)
    \(\begin{cases} x | y \Then y = kx\\ y | x \Then x = ty \end{cases}\)
    Тогда \(y=kty \Then kt = 1\). Значит, \(k\) и \(t\) обратимы. Значит, \(x=ty, t \in K^* \Then x \~ y\) по определению.
  \end{itemize}

  \end{solution}
\i
  Отношение \(\sim\) является отношением эквивалентности.
  \begin{solution}

  \begin{enumerate}
  \def\labelenumii{\arabic{enumii}.}
  \i
    \(x\~x\), т. к. \(\exists 1 \in K^*: x=1x\)
  \i
    \(x \~ y \Then \exists r \in K^*: x = ry \Then y = r^{-1}x \Then y \~ x\)
  \i
    \(x \~ y, y \~ z \Then \begin{cases} \exists r_1 \in K^*: x=r_1 y\\ \exists r_2 \in K^*: y=r_2 z \end{cases} \Then x = \ubrace{r_1 r_2}{\in K^* \text{, т. к. }(r_1 r_2)^{-1} = r_2^{-1} r_1^{-1}} z \Then x \~ z\)
  \end{enumerate}

  \end{solution}
\end{enumerate}

\begin{problem}[11 (2.5)]
Если $k \in \Z$, то $z= a+bu \in D$ делится на $k$ тогда и только тогда, когда $a$ и $b$ делятся на $k$.
\end{problem}

\begin{solution}

\begin{itemize}
\i
  \(\Then:\)
  \(\begin{cases} a \Divby k \\ b \Divby k \end{cases} \Then \begin{cases} a = ka' \\ b = kb' \end{cases} \Then z = a+bu = ka'+kb'u=k(a'+b'u) \Then z \Divby k\)
\i
  \(\Then\):
  Пусть \(z = a + bu = ka' + kb'u\). Тогда \((a - ka') = u(b - kb')\).

  Обе части целые \(\Then\) нули, потому что \(u\) не рациональное.

  Отсюда \(\begin{cases} a = ka' \\ b = kb' \end{cases} \Then \begin{cases} a \Divby k \\ b \Divby k \end{cases}\).
\end{itemize}

\end{solution}

\begin{problem}[12(2.9 $\When$)]
$K$ — евклидово кольцо. Верно ли, что для $a \ne 0, b \in K^*$ выполнено равенство $N(ab) = N(a)$?
\end{problem}

\begin{solution}
\(b \in K^* \Then N(a) \le N(ab) \le N(abb^{-1}) = N(a)\)
\end{solution}

\begin{problem}[13 (3.2)]
Для $u=i,\omega$ и простого целого числа $p \leq 40$ выясните, существует ли $z \in \Z[u]$ с $N(z)=p$. Сформулируйте гипотезу о том, какие простые целые числа являются простыми в $\Z[u]$. 
\end{problem}

\begin{solution}

Выпишем все варианты \(a, b\) с нормой \(\le 40\).

\bf{Зам.} Можно опустить перебор по \(ka', kb'\) при \(k > 1\), потому что тогда обе нормы делятся на \(k^2\).

\bf{Зам.} Можно брать только натуральные, т. к. для $\Z[i]$ норма не поменяется вообще, а для $\Z[\omega] \ N =  a^2 + ab + b^2 = a^2 - a (a+b) + (a+b)^2$, т. е. норма элемента $a - b \omega$ равна норме элемента $a + (a+b) \omega$, а такие мы уже перебрали, поскольку $a+b$ --- натуральное. Для $a <0$ симметрично.

\begin{longtable}[]{@{}llll@{}}
\toprule
a & b & \(\mathbb{Z}[i], N = a^2+b^2\) & \(\mathbb{Z}[\omega], N = a^2-ab+b^2\)\tabularnewline
\midrule
\endhead
1 & 1 & 2 & 1\tabularnewline
1 & 2 & 5 & 3\tabularnewline
1 & 3 & 10 & 7\tabularnewline
1 & 4 & 17 & 13\tabularnewline
1 & 5 & 26 & 21\tabularnewline
1 & 6 & 37 & 31\tabularnewline
2 & 2 & - & -\tabularnewline
2 & 3 & 13 & 7\tabularnewline
2 & 4 & - & -\tabularnewline
2 & 5 & 29 & 19\tabularnewline
2 & 6 & - & -\tabularnewline
2 & 7 & 53 & 39\tabularnewline
3 & 3 & - & -\tabularnewline
3 & 4 & 25 & 13\tabularnewline
3 & 5 & 34 & 19\tabularnewline
3 & 6 & - & -\tabularnewline
3 & 7 & 58 & 37\tabularnewline
4 & 4 & - & -\tabularnewline
4 & 5 & 41 & 21\tabularnewline
4 & 6 & - & -\tabularnewline
4 & 7 & 65 & 37\tabularnewline
5 & 5 & - & -\tabularnewline
5 & 6 & 61 & 31\tabularnewline
5 & 7 & 74 & 39\tabularnewline
6 & 6 & - & -\tabularnewline
\bottomrule
\end{longtable}

Пользуемся №3.1б (№9 exam\_5-6): Пусть $p$ --- простое целое, \(\forall z \in \Z[u]: N(z)\ne p \Then\) $p$ неразложим в $\Z[u]$.

Выпишем все простые числа \(\le 40\) и вычеркнем те, которые являются нормой. Берём оставшиеся.

\begin{table}[H]
\centering
\begin{tabular}{llllll}
\multicolumn{3}{c}{$\Z[i]$} & \multicolumn{3}{c}{$\Z[\omega]$} \\
        & 2      & \x     & \y       & 2        &          \\
\y      & 3      &        &          & 3        & \x       \\
        & 5      & \x     & \y       & 5        &          \\
\y      & 7      &        &          & 7        & \x       \\
\y      & 11     &        & \y       & 11       &          \\
        & 13     & \x     &          & 13       & \x       \\
        & 17     & \x     & \y       & 17       &          \\
\y      & 19     &        &          & 19       & \x       \\
\y      & 23     &        & \y       & 23       &          \\
        & 29     & \x     & \y       & 29       &          \\
\y      & 31     &        &          & 31       & \x       \\
        & 37     & \x     &          & 37       & \x        
\end{tabular}
\end{table}

Гипотеза:
\begin{itemize}
\i у \(\Z[i]\) --- \(4k+3\)
\i у \(\Z[\omega]\) --- \(3k+2\)
\end{itemize}
\end{solution}

\begin{problem}[14 (3.9)]
\end{problem}

\begin{solution}

\begin{enumerate}
\def\labelenumi{\alph{enumi})}
\tightlist
\i
  \(0 \subset K, K \subset K\) --- идеалы. Они называются \bf{тривиальными}.

  \begin{itemize}
  \tightlist
  \i
    \(\{0\}\):

    \begin{enumerate}
    \def\labelenumii{\arabic{enumii}.}
    \tightlist
    \i
      Тривиальная группа по сложению:

      \begin{itemize}
      \tightlist
      \i
        Ассоциативность наследуется
      \i
        \(0\) --- нейтральный элемент, т. к. \(0+a=a+0=0\ \forall a \in \{0\}\)
      \i
        \(0^{-1} = 0 = -0\)
      \end{itemize}
    \i
      Замкнутость относительно умножения: \(\forall a \in K \Have  0a=0 \in \{0\}\)
    \end{enumerate}
  \i
    \(K\):

    \begin{enumerate}
    \def\labelenumii{\arabic{enumii}.}
    \tightlist
    \i
      Тривиальная группа по сложению:

      \begin{itemize}
      \tightlist
      \i
        Ассоциативность наследуется
      \i
        \(0\) --- нейтральный элемент, т. к. \(0+a=a+0=a\ \forall a \in K\)
      \i
        \(a^{-1} = -a \in K\)
      \end{itemize}
    \i
      Замкнутость относительно умножения:
      \(\forall a \in K, b \in I=K \Have ab \in I=K\) --- по свойству кольца
    \end{enumerate}
  \end{itemize}
\i
  \((a) = \{ax \mid x\in K\}\) --- \bf{главный идеал} или \bf{идеал, порождённый одним элементом}

  \begin{enumerate}
  \def\labelenumii{\arabic{enumii}.}
  \tightlist
  \i
    Подгруппа по сложению:

    \begin{itemize}
    \tightlist
    \i
      \(ax_1+ax_2=a(x_1+x_2)\in(a)\) --- замкнутость относительно сложения
    \i
      Ассоциативность наследуется
    \i
      \(0\) --- нейтральный элемент: \(ax+0=0+ax=ax\)
    \i
      \(ax+a(-x)=a(x-x)=a\cdot 0=0\)
    \end{itemize}
  \i
    Замкнутость относительно умножения:
    \(\forall b \in K, ax \in (a) \Have b\cdot ax = bx \cdot a \in (a)\)
  \end{enumerate}
\i
  \((a_1,\ldots,a_n) = \{a_1x_1+\ldots+a_nx_n \mid x_1,\ldots,x_n \in K\}\) --- \bf{конечно-порождённый идеал}, то есть идеал, порождённый конечным количеством элементов.

  \begin{enumerate}
  \def\labelenumii{\arabic{enumii}.}
  \tightlist
  \i
    Подгруппа по сложению:

    \begin{itemize}
    \tightlist
    \i
      \((a_1x_1+\dots+a_nx_n)+(a_1y_1+\dots+a_ny_n) = a_1(x_1+y_1)+\dots+a_n(x_n+y_n) \in I\) --- замкнутость относительно сложения
    \i
      Ассоциативность наследуется
    \i
      \(0 = a_1\cdot 0+\dots+a_1\cdot0\) --- нейтральный элемент: \(ax+0=0+ax=ax\)
    \i
      \((a_1x_1+\dots+a_nx_n)+(a_1(-x_1)+\dots+a_n(-x_n))=0\)
    \end{itemize}
  \i
    Замкнутость относительно умножения:
    \(\forall y \in K \ y\cdot (a_1x_1+\dots+a_nx_n) = a_1(x_1y)+\dots+a_n(x_ny) \in I\)
  \end{enumerate}
\end{enumerate}

\end{solution}

\begin{problem}[15(3.11)]
a) Докажите, что $(a) \subset (b)$ тогда и только тогда, когда $b \mid a$.

b) Докажите, что $a \sim b$ тогда и только тогда, когда $(a)=(b)$.
\end{problem}

\begin{solution}

\begin{enumerate}
\def\labelenumi{\alph{enumi})}
\i
  \begin{itemize}
  \tightlist
  \i
    \(\When:\) \(b | a \Then \exists c: a=cb \Then ka=(kc)b \Then (a) \subset (b)\)
  \i
    \(\Then:\) \((a) \subset (b) \Then a \in (b) \Then a=cb \Then b|a\)
  \end{itemize}
\i
  \begin{itemize}
  \tightlist
  \i
    \(\Then:\) \(a \~ b \Then \begin{cases} a|b \\ b|a \end{cases} \Then (a) \subset (b) \subset (a) \Then (a) = (b)\)
  \i
    \(\When:\) \((a) = (b) \Then \begin{cases} a|b \\ b|a \end{cases} \Then a \~ b\)
  \end{itemize}
\end{enumerate}

\end{solution}

\begin{problem}[16(3.12)]
Пусть $I,J \subset K$ --- идеалы. \bf{Сумма} $I+J = \{x+y \mid x \in I,y \in J\}$ и \bf{пересечение} $I \cap J$ идеалов являются идеалами. 
\end{problem}

\begin{solution}

\begin{enumerate}
\def\labelenumi{\alph{enumi})}
\i
  \begin{enumerate}
  \def\labelenumii{\arabic{enumii}.}
  \i
    \begin{itemize}
    \tightlist
    \i
      \((x_1+y_1)+(x_2+y_2) = \ubrace{(x_1+x_2)}{\in I}+\ubrace{(y_1+y_2)}{\in J} \in I+J\)
    \i
      Ассоциативность следует.
    \i
      0 --- нейтральный.
    \i
      \((x+y)+\ubrace{(-x-y)}{\in I+J} = (x-x)+(y-y)=0\) --- обратный
    \end{itemize}
  \i
    \(\forall a \in K \Have a(x+y)=\ubrace{ax}{\in I}+\ubrace{ay}{\in J} \in I+J\)
  \end{enumerate}
\i
  \begin{enumerate}
  \def\labelenumii{\arabic{enumii}.}
  \i
    \begin{itemize}
    \tightlist
    \i
      \(x, y \in I \cap J \Then \begin{cases} x, y \in I \\ x, y \in J \end{cases} \Then \begin{cases} x + y \in I \\ x + y \in J \end{cases} \Then x+y \in I\cap J\)
    \i
      Ассоциативность следует.
    \i
      0 --- нейтральный
    \i
      \(x \in I\cap J \Then \begin{cases} x \in I \\ x\in J \end{cases} \Then \begin{cases} x^{-1} \in I \\ x^{-1} \in J \end{cases} \Then x^{-1} \in I\cap J\) --- обратный
    \end{itemize}
  \i
    \(\forall a \in K\ \forall x \in I\cap J \Have  \begin{cases} x\in I \\ x\in J \end{cases} \Then  \begin{cases} ax\in I \\ ax\in J \end{cases} \Then  ax \in I \cap J\)
  \end{enumerate}
\end{enumerate}

\end{solution}

\begin{problem}[17(3.15)]
Пусть $K \neq 0$. Докажите, что $K$ является полем тогда и только тогда, когда $K$ не содержит нетривиальных идеалов.
\end{problem}

\begin{solution}

\begin{itemize}
\i
  \(\Then:\)
  Пусть $K$ --- поле, \(I \subset K\) --- идеал.

  \begin{itemize}
  \tightlist
  \i
    \(x = 0 \Then (x) = \{0\}\) --- тривиальный идеал.
  \i
    \(\forall x \in I, x \ne 0, \ x\) обратим по свойству поля, значит, \(I \supset (x) = (1) = K\).
  \end{itemize}
\i
  \(\When:\)
  Пусть $K$ --- коммутативное кольцо без нетривиальных идеалов. Пусть \(x \in K, x \ne 0,\) --- произвольный элемент. Тогда \((x) \ne \{0\}\). Значит, поскольку у нас нет нетривиальных идеалов, \((x) = K\).

  В частности, \(1 \in (x) = K \Then \exists x^{-1}\), т. е. элемент $x$ обратим.

  В силу произвольности $x$, любой ненулевой элемент обратим \(\Then\) $K$ --- поле (в $K$ \(\ge 2\) элементов, т. к. \(0 \in K\), и мы брали \(0\ne x \in K\)).
\end{itemize}

\end{solution}

\begin{problem}[18(4.1)]
Верно ли, что при гомоморфизме колец $\varphi: K \to L$ 
a) образ идеала $I \subset K$ является идеалом в $L$; b) прообраз идеала $J \subset L$ является идеалом в $K$?
\end{problem}
\begin{solution}
\begin{enumerate}
\def\labelenumi{\alph{enumi})}
\i
  Неверно. Контрпример: \(\phi:\Z \to \Q, \phi(x)=x\) --- поэлементное вложение.

  \(I=\Z\) в \(\Z\) --- тривиальный идеал. Но \(\phi(I)=\Z\) --- не идеал в \(\Q\), ибо, например, \(\ubrace{\frac{1}{2}}{\in \Q} \cdot \ubrace{1}{\in \Z} = \frac{1}{2} \not\in I\).
\i
  Верно. Пусть \(J\) --- идеал в L. \(\phi^{-1}(J) = \{a \in K : \phi(a) \in J\}\).

  \(\forall a, b \in \phi^{-1}(J): 
	\begin{cases} 
		a+b \in \phi^{-1}(J), \text{т. к. } \phi(a+b) = \phi(a)+\phi(b) \in J \\
		\exists 1 \in \phi^{-1}(J), \text{т. к. } \phi(1)=1 \text{ по свойству гомоморфизма}\\
		\exists a^{-1} \in \phi^{-1}(J), \text{т. к. }\phi(a^{-1}) = (\phi(a))^{-1} \in J 
	\end{cases}\)

  \(\forall x \in K,\ a \in \phi^{-1}(J) \Have \phi(ax)=\phi(a)\phi(x) \in J\).

  Значит, \(\phi^{-1}(J)\) --- действительно идеал.
\end{enumerate}
\end{solution}

\begin{problem}[19(4.2)]
\end{problem}

\begin{enumerate}
\def\labelenumi{\alph{enumi})}
\tightlist
\i
  Всегда ли факторкольцо коммутативного кольца является коммутативным кольцом?
  \begin{solution}

  \begin{itemize}
  \tightlist
  \i
    Ассоциативность по сложению --- из ассоциативности коммутативного кольца.
  \i
    \(0 \in K\) --- ноль в \(K\) \(\Then\) \(0+I=I\) --- ноль в \(K^*\): \((I)(a+I) = (a+I)(I) = aI+I^2 = I\).
  \i
    Обратный по сложению: \((a+I)+(-a+I) = (-a+I)+(a+I) = I\).
  \i
    Дистрибутивность: \((a+I)(b+I+c+I) = (ab+I)+(ac+I)\).
  \i
    \(1 \in K\) --- единица в \(K\) \(\Then\) \(1+I\) --- единица в \(K^*\): \((1+I)(a+I) = (a+I)(1+I) = a+I+aI+I^2 = a+I\).
  \i
    Ассоциативность по умножению --- из ассоциативности коммутативного кольца.
  \i
    \((a+I)(b+I) = ab+aI+bI+II=ab+I=ba+I=ba+bI+aI+II=(b+I)(a+I)\) --- коммутативность.
  \end{itemize}

  \end{solution}
\i
  Имеется \bf{канонический} гомоморфизм \(\varphi: K \to K/I\), который переводит \(a \mapsto a+I\).
  \begin{solution}
  Проверим свойства гомоморфизма:

  \begin{itemize}
  \tightlist
  \i
    \(\phi(a)+\phi(b)=a+I+b+I= (a+b)+I=\phi(a+b)\)
  \i
    \(\phi(a)\phi(b) = (a+I)(b+I) = ab+aI+bI+II = ab+I = \phi(ab)\)
  \i
    \(\phi(1) = 1+I = 1_{\sfrac{K}{I}}\)
  \end{itemize}

  \end{solution}
\end{enumerate}

\begin{problem}[20(4.5)]
Пусть $K$ --- область целостности. Идеал $(x)$ является простым тогда и только тогда, когда $x$ прост.
\end{problem}

\begin{solution}
\((x)\) --- простой \(\Bydef\) если \(ab \in (x)\), то \(\begin{sqcases} a \in (x) \\ b \in (x) \end{sqcases}\)

\(x\) --- простой \(\Bydef\) если \(ab \Divby x\), то \(\begin{sqcases} a \Divby x \\ b \Divby x \end{sqcases}\)

Но \(ab \in (x) \Iff ab \Divby x\) (ибо \((x) = \{cx \mid c \in K\}\) по определению, и \(ab \in K\)).
\end{solution}

\begin{problem}[21(4.6) (Lecture\_all.pdf теор. 3.2)]
Пусть $K$ --- область целостности. Нетривиальный идеал $I$ является максимальным тогда и только тогда, когда $K/I$ поле.
\end{problem}

\begin{solution}
Знаем (№17): \(\sfrac{K}{I}\) --- поле \(\Iff\) в \(\sfrac{K}{I}\) нет нетривиальных идеалов.

% Пусть \(\sfrac{K}{I}\) --- поле, пусть \(\exists J: I \subset J \subset K\) --- нетривиальный идеал. Подействуем на него 
Рассмотрим канонический гомоморфизм \(f: K \to \sfrac{K}{I}\). Заметим: $f(I) = \{0\}, f(K) = \sfrac{K}{I}$.

\bf{Лемма.} Пусть \(f: K \to L\) --- гомоморфизм колец, \(I \subset K, J \subset L\) --- идеалы. Тогда a) \(f(I)\) --- идеал в \(f (K)\), b) \(f^{-1} (J)\) --- идеал в K.

\begin{solution}
a) Пусть $x \in f(I), y \in f(K)$. Тогда найдутся такие $x'$ и $y'$, где $x' \in I, x = f(x'), y' \in K, y = f(y')$.
    Имеем: $xy = f(x')f(y') = f (x' y') \in F(I)$, так как $x' y' \in I$.
    
b) (можно сослаться на №18(б)) Пусть теперь $x \in f^{-1}(J), y \in K$. Тогда $f(xy) = f(x)f(y) \in J$, следовательно, $xy \in f^{-1}(J)$.
\end{solution}

% Из Леммы следует, что в \(\sfrac{K}{I}\) существует нетривиальный идеал \(\Iff\) существует идеал в \(K\), содержащий \(I\):
\begin{itemize}
\i $\When:$ Пусть $\sfrac{K}{I}$ --- поле, но идеал $I$ не максимальный. Тогда $\exists$ нетривиальный идеал $J \subset K: I \subset J$.
По пункту a Леммы $f(J)$ --- идеал в $f(K) = \sfrac{K}{I}$. При этом $I \subset J \Then \{0\} = f(I) \subset f(J)$. $f(J)$ нетривиален, ибо он не $\{0\}$ (иначе $J=I$) и не $\sfrac{K}{I}$ (иначе $J=K$). То есть, получили в поле нетривиальный идеал. Противоречие с тем, что это поле.

\i $\Then:$ Пусть идеал $I$ максимален, но $\sfrac{K}{I}$ --- не поле. Тогда должен быть нетривиальный идеал $L \subset \sfrac{K}{I}$. Его прообраз $f^{-1}(L)$ по пункту b Леммы --- идеал в $K$. При этом $L \supset \{0\} \Then f^{-1}(L) \supset f^{-1}(\{0\}) = I$. Он нетривиален, ибо его прообраз не $I$ (иначе $J=\{0\})$ и не $K$ (иначе $J = K/I$). Т. е., получили нетривиальный идеал в $K$, содержащий $I$ --- противоречие с максимальностью $I$.
\end{itemize}

\end{solution}

\begin{problem}[22(4.7)]
Пусть $K$ --- область целостности. Нетривиальный идеал $I$ является простым тогда и только тогда, когда $K/I$ область целостности.
\end{problem}

\begin{solution}

\begin{itemize}
\tightlist
\i
  \(\Then:\)
  Пусть I --- простой, но \(\sfrac{K}{I}\) --- не область целостности. Тогда \(\exists a, b \in K\setminus I: (a+I)(b+I) = ab+I = 0+I = 0_{\sfrac{K}{I}}\). Но тогда должно быть \(ab \in I\), т. е. идеал не простой (вспомним, что брали $a, b \in K\setminus I$). Противоречие.
\i
  \(\When:\)
  Пусть I непростой, но $\sfrac{K}{I}$ --- область целостности. Тогда \(\exists a, b: ab \in I\), но \(a, b \not\in I\). Рассмотрим \((a+I)(b+I) = ab + I \ubrace{=}{ab \in I} I = 0_{\sfrac{K}{I}}\), т. е. $\sfrac{K}{I}$ --- не область целостности.
\end{itemize}

\end{solution}

\begin{problem}[23(5.1, 5.2)]
Пусть $K$ --- область целостности. Рассмотрим  множество пар $\tilde{K}=\{a,b\}$ элементов кольца $K$, где $b\neq 0$. На этом множестве введем отношение следующим образом: $\{a,b\} \sim \{c,d\}$, если $ad=bc$.

a) Докажите, что $\{a,b\} \sim \{ac,bc\}$. 
b) Докажите, что это отношение эквивалентности.

Элемент множества классов эквивалентности $F = \Quot(K)$ будем записывать как $\frac{a}{b}$ или $ab^{-1}$. Введем операции сложения и умножения на $F = \Quot(K)$:

$$\frac{a}{b} + \frac{c}{d} = \frac{ad+bc}{bd},$$
$$\frac{a}{b} \cdot \frac{c}{d} = \frac{ac}{bd}.$$

Докажите, что

c) сложение и умножение корректно определено;
d) $F$ является коммутативным кольцом;
e) $F$ является полем;
f) существует инъекция $K \to F$.

\end{problem}

\begin{solution}

\begin{enumerate}
\def\labelenumi{\alph{enumi})}
\i
  \(a \cdot bc = b \cdot ac\) --- из коммутативности.
\i
  \begin{itemize}
  \tightlist
  \i
    \(\{a, b\} \~ \{a, b\}\), т. к. \(ab=ab\)
  \i
    \(\{a, b\} \~ \{c, d\} \Iff ad=bc \Iff cb=da \Iff \{c, d\} \~ \{a, b\}\)
  \i
    \(\{a, b\} \~ \{c, d\} \~ \{e, f\} \Then \begin{cases} ad=bc\\cf=de\end{cases} \Then \begin{cases} adf=bcf\\bcf=bde\end{cases} \Then adf=bde \Then af=be \Then \{a, b\} \~ \{e, f\}\)
  \end{itemize}
\i
  Корректность определения означает, что множество замкнуто относительно операции, и что результат её всегда определён.

  \begin{itemize}
  \tightlist
  \i
    Сложение:

    \begin{itemize}
    \tightlist
    \i
      \(\frac{ad+bc}{bd} \in \Quot(K)\), т. к. \(ad+bc \in K\) и \(bd \in K\) по свойствам кольца \(K\).
    \i
      У \(\frac{ad+bc}{bd}\) \(bd \ne 0\), т. к. \(b \ne 0\) и \(d \ne 0\).
    \end{itemize}
  \i
    Умножение:

    \begin{itemize}
    \tightlist
    \i
      \(\frac{ac}{bd} \in \Quot(K)\), т. к. \(aс \in K\) и \(bd \in K\) по свойствам кольца \(K\).
    \i
      У \(\frac{ac}{bd}\) \(bd \ne 0\), т. к. \(b \ne 0\) и \(d \ne 0\).
    \end{itemize}
  \end{itemize}
\i
  \begin{itemize}
  \tightlist
  \i
    Это кольцо:

    \begin{itemize}
    \tightlist
    \i
      \(\frac{a}{b} + (\frac{c}{d} + \frac{e}{f}) = \frac{a}{b} + (\frac{cf+de}{df}) = \frac{adf+bcf+bde}{bdf} = \frac{ad+bc}{bd}+\frac{e}{f} = (\frac{a}{b} + \frac{c}{d}) + \frac{e}{f}\)
    \i
      Ноль --- элемент класса эквивалентности \(\{\frac{0}{a} \mid a \ne 0\}\). Возьмём \(0_F := \frac{0}{1}\). Тогда \(\frac{a}{b} + \frac{0}{1} = \frac{0}{1} + \frac{a}{b} = \frac{0}{1}\)
    \i
      Для \(\frac{a}{b}\) обратный по сложению \(\frac{-a}{b}\): \(\frac{a}{b}+\frac{-a}{b} = \frac{a-a}{b} = \frac{0}{b}\)
    \i
      \(\frac{a}{b} (\frac{c}{d} + \frac{e}{f}) = \frac{a}{b} \cdot \frac{cf+de}{df} = \frac{acf+ade}{bdf} = \frac{ac}{bd}+\frac{ae}{bf} = \frac{a}{b}\cdot \frac{c}{d}+\frac{a}{b} \cdot \frac{e}{f}\)
    \end{itemize}
  \i
    Оно ассоциативно по умножению: \(\frac{a}{b} (\frac{c}{d} \cdot \frac{e}{f}) = \frac{a}{b} (\frac{ce}{df} ) = \frac{ace}{bdf} = (\frac{ac}{bd}) \frac{e}{f} = (\frac{a}{b} \cdot \frac{c}{d}) \cdot \frac{e}{f}\)
  \i
    Единица --- элемент класса эквивалентности \(\{\frac{a}{a} \mid a \ne 0\}\). Возьмём \(1_F = \frac{1}{1}\). Тогда \(\frac{a}{b} \frac{1}{1} = \frac{1}{1} \frac{a}{b} = \frac{a}{b}\)
  \i
    Оно коммутативно: \(\frac{a}{b} \frac{c}{d} = \frac{ac}{bd} = \frac{ca}{db} = \frac{c}{d} \frac{a}{b}\)
  \end{itemize}
\i
  \begin{itemize}
  \tightlist
  \i
    Каждый ненулевой элемент имеет обратный по умножению \((\frac{a}{b})^{-1} = \frac{b}{a}\): \(\frac{a}{b} \frac{b}{a} = \frac{ab}{ba} = \frac{ab}{ab} = \frac{1}{1}\).
  \i
    Элементов \(\ge 2\), т. к. \(0 \ne 1\).
  \end{itemize}
\i
  Возьмём \(\phi: K \to F, \phi(a) = \frac{a}{1}\).

  Это гомоморфизм: 
  $\begin{cases} 
    \frac{ab}{1} = \phi(ab) = \phi(a)\phi(b) = \frac{a}{1}\frac{b}{1} = \frac{a\cdot b}{1\cdot 1} = \frac{ab}{1}\\
    \frac{a+b}{1} = \phi(a+b) = \phi(a)+\phi(b) = \frac{a}{1}+\frac{b}{1} = \frac{a\cdot 1+1 \cdot b}{1\cdot 1} = \frac{a+b}{1}
  \end{cases}$

  Это инъекция: \(\forall a \ne b \in K \Have \phi(a) = \frac{a}{1} \ne \frac{b}{1} = \phi(b)\), ибо \(\frac{a}{1} - \frac{b}{1} = \frac{a-b}{1} \ne 0_F\).
\end{enumerate}

\end{solution}

\begin{problem}[24(6.1)]
\bf{Признак неприводимости Эйзенштейна}

Пусть $f(x)$ --- многочлен с целыми коэффициентами и существует такое простое число $p$, что:

\begin{enumerate}
\i старший коэффициент $f(x)$ не делится на $p$;
\i все остальные коэффициенты $f(x)$ делятся на $p$;
\i свободный член $f(x)$ не делится на $p^2$.
\end{enumerate}
Тогда многочлен $f(x)$ неприводим над полем рациональныx чисел.
\end{problem}

\begin{solution}
Пусть, напротив \(f\) не неприводим над \(\Q\). Тогда существуют два таких многочлена \(g,h\in\Q[x]\), что
\(f=\tilde{g}\tilde{h}\) и \(\deg{\tilde{g}}, \deg{\tilde{h}} > 0\).
Положим \(d_1 = (\text{НОД знаменателей коэффициентов }\tilde{g})\) и \(d_2 = (\text{НОД знаменателей коэффициентов }\tilde{h})\). Тогда есть разложение \(d_1 d_2 f = g h\), где \(g, h \in \Z[x]\) ---- многочлены, полученные домножением \(\tilde{g},\tilde{h}\) на \(d_1,d_2\) соответственно. Так как \(g\) и \(h\) примитивны, то их произведение примитивно, а значит, \(d_1 d_2 = 1\) и верно просто \(f = gh\).

Заметим, что \(f_0 = g_0 h_0\). Так как \(f_0\) не делится на \(p^2\), то одно из чисел \(g_0, h_0\) не делится на \(p\). Пусть для определенности \(h_0\) не делится на \(p\), тогда \(g_0\) делится на \(p\). Пусть число \(k > 0\) таково, что \(g_0, \ldots, g_{k-1}\) делятся на \(p\), а \(g_k\) не делится на \(p\). По формуле коэффициентов для произведения многочленов \(f_k = g_k h_0 + g_k h_1 + \ldots\). Если \(k < \deg f\), то по модулю \(p\) имеем \(0 = g_k h_0\), где \(g_k, h_0\) не делятся на \(p\). Значит, \(k = \deg f > \deg g\). Это означает, что все коэффициенты \(g\) делятся на \(p\), то есть \(g\) не примитивен.
Противоречие.

%\bf{Решение от Ильинского}
%Пусть $f$ приводим над $\Q$, тогда он представляется в виде произведения двух многочленов из $\Z[x]$ ненулевой степени $g$ и $h$ (воспользовались №5-6.24). Заметим сразу, что оба многочлена имеют степень, меньше, чем n. 

%Если g или h делятся на p, то и f делится на p, что противоречит условию. При этом ровно один из коэффициентов g_0 или h_0 не делится на p, так как иначе g_0h_0 делился бы на p^2 или не делился бы на p. Будем считать, что g_0 не делится на p, а h_0,...,h_{k-1} делятся на p, h_k делится на p. Тогда коэффициент f_k = g_0h_k+g_1h_{k-1}+.... не делится на p. Значит, k=n. Но мы уже договорились, что k < n - получаем противоречие. 

%Пусть не так, и он приводим в $\Q$. По №5-6.24а, это эквивалентно приводимости в $\Z[x]$.
%Разложим $f$ в $\Q$: $f = \tilde{g} \tilde{h}$, где $\tilde{g}, \tilde{h} \in \Q[x]$. Положим $Q =Q_1 \cdot Q_2$, где $Q_1, Q_2=$ НОД(знаменателей модулей коэффициентов в несократимой записи $\tilde{g}, \tilde{h}$ соответственно). Тогда получили разложение $Qf = gh$, где $g, h \in \Z[x]$. Распишем: $f_nx^n+\dots+f_1x+f_0=f(x)=g(x)h(x) = (g_kx^k+\dots+g_1x+g_0)(h_mx^m+\dots+h_1x+h_0), 0 < \deg g, \deg h < n$. Возьмём всё по модулю $p$ (если мы утверждаем, что у нас равенство выполняется в $\Z$, то оно должно выполняться и для любого натурального модуля). 

%* $Q \not\Divby p:$
%	$Тогда $\ol{f}(x) \expl{=}{\text{т. к. другие члены делятся на }p} \ol{f}_nx^n$. $f(x)$ состоит из одного монома, а произведение двух многочленов будет одним мономом $\Iff$ оба этих многочлена тоже мономы. Отсюда $\ol{f}(x) = \ol{g}(x)\ol{h}(x) = (g_kx^k)(h_m x^m)$. Рассмотрим свободный член. Если $k, m > 1$, то $Qa_0 = \ubrace{g_0}{\Divby p} \ubrace{h_0}{\Divby p} \Divby p^2$ (свободные члены $g(x)$ и $h(x)$ делятся на $p$, т. к. они зануляются, когда мы берём по модулю $p$) $\Then$ $Q \Divby p$ --- противоречие с $Q \not\Divby p$.
%* $Q \Divby p$:
%	Тогда $Q = p^\alpha t, t \not\Divby p$. По модулю $p$: $0 = (g_kx^k)(h_mx^m) \Then$ все коэффициенты $g_k$ и $h_m$ делятся на $p \Then$ на самом деле $Q = p^{\alpha-1} t$ (раз у многочленов $g, h$ все коэффициенты делятся на p, то НОД можно сократить на p). Противоречие с определением $Q$.
\end{solution}

\begin{problem}[25(указано 6.2, но на самом деле в нём точно такого пункта нет)]
Многочлен $x^n - p$ ($p$ --- простое число) неприводим над $\Q$.
\end{problem}

\begin{solution}
По критерию Эйзенштейна: \(1 \not\Divby p, -p \Divby p, -p \not\Divby p^2\), где p --- простое.
\end{solution}

\begin{problem}[26(6.3)]
Характеристика поля --- простое число.
\end{problem}

\begin{solution}
Если $\Char F = 1$, то $1 = 0$, поле из одного элемента, что не является полем по нашей договорённости.

Если $\Char F = mn, m, n \in \N, m, n > 1$, то $(\ubrace{1+\dots+1}{m})(\ubrace{1+\dots+1}{n}) \ubrace{=}{\text{по дистрибутивности}} \ubrace{1+\dots+1}{m\cdot n} = 0$. Противоречие с минимальностью $\Char F$.
\end{solution}

\begin{problem}[27(6.4)(Lecture\_all.pdf №6.2(3)]
Если существует нетривиальный гомоморфизм полей $\phi: F \to K$, то $\mathrm{char}(F) = \mathrm{char}(K)$.
\end{problem}

\begin{solution}
Гомоморфизм нетривиален $\Then$ по №29 он является инъекцией, а у инъекции \(\Ker \phi = \{0\}\) по лемме из №29.
Так как \(\phi(1) = 1\), имеем \(\phi(\ubrace{1+\dots+1}{m}) = \ubrace{1+\dots+1}{m}\). 

Если \(\ubrace{1+\dots+1}{m} = 0\) в F, то по свойству гомоморфизма и в $K$ тоже. Получили $\Char(K) \le \Char(F)$.

Т. к. \(\Ker \phi = \{0\}\), то только 0 переходит в 0, т. е. получив сложением единичек $0$ в $K$ знаем, что в $F$ тоже $0$. Отсюда $\Char(F) \le \Char(K) $.

% Т. к. \(\Ker \phi = \{0\}\), то \(\ubrace{1+\dots+1}{m} = 0\) в \(K\) и \(F\) одновременно. Следовательно, \(\Char F = \Char K\).
\end{solution}

\begin{problem}[28(6.5)]
Любое конечное поле имеет положительную характеристику.
\end{problem}

\begin{solution}
Пусть F конечно, а \(\Char F = 0\). Тогда \(\ubrace{1+\dots+1}{k}\) для любого $k$ будет давать элемент поля, не совпадающий с предыдущими (иначе \(\Char\) была бы конечна).

Получается, что F бесконечно. Противоречие.
\end{solution}

\begin{problem}[29(№6.7)]
Нетривиальный гомоморфизм полей $\varphi: F \to L$ является инъекцией.
\end{problem}

\begin{solution}

\bf{Лемма.} \(\phi: F \to L\) -- инъекция \(\Iff \Ker \phi = \{0\}\).
\begin{solution}
\begin{itemize}
\i
  \(\Then:\)
  \(\phi\) -- инъекция \(\Bydef \forall a \ne b \in F \Have \phi(a) \ne \phi(b)\).

  \(\Ker \phi = \{a \in F: \phi(a) = 0_L\}\).

  Имеем \(\phi(0)=0\) по свойству гомоморфизма, тогда по инъективности \(\forall a \ne 0 \Have \phi(a) \ne \phi(0)=0\), т. е. \(\Ker \phi = \{0\}\).
\i
  \(\When:\)
  Пусть не так. \(\Ker \phi \ne \{0\} \Then \exists 0 \ne a \in \Ker \phi\). Тогда \(\forall b \in K \Have \phi(b+a) = \phi(b)+\phi(a) = \phi(b)\) --- нарушение инъективности.
\end{itemize}

\end{solution}

\bf{Лемма.} \(\Ker \phi\) --- идеал в F
\begin{solution}
$\Ker \phi$ --- подгруппа по сложению --- простая проверка.

$\forall a, b \in \Ker \phi \Have ab \in \Ker \phi$, т. к. $\phi(ab) = \phi(a)\phi(b)=0\cdot0=0$ --- замкнутость относительно умножения.

\(\forall a \in F, x \in \Ker \phi \Have \phi(ax)=\phi(a)\phi(x)=\phi(a) \cdot 0 = 0 \Then ax \in \Ker \phi\)
\end{solution}

В поле F идеал \(I=\begin{cases} \{0\} \\ F \end{cases}\), т. е. \(\Ker \phi = \begin{cases} \{0\} \\ F \text{ --- невозможно} \end{cases}\)

(в последнем случае гомоморфизм тривиален, но у нас нетривиальный по условию).

\end{solution}

\begin{problem}[30(№6.8)]
$K$ образует линейное пространство над $F$.
\end{problem}

\begin{solution}
Проверка свойств. Свойства линейного пространства следуют из аксиом поля.

\begin{enumerate}
\i $\mathbf{x} + \mathbf{y} = \mathbf{y} + \mathbf{x}$, для любых $\mathbf{x}, \mathbf{y}\in V$ (''коммутативность сложения'');
\i $\mathbf{x} + (\mathbf{y} + \mathbf{z}) = (\mathbf{x} + \mathbf{y}) + \mathbf{z}$, для любых $\mathbf{x}, \mathbf{y}, \mathbf{z} \in V$ (''ассоциативность сложения'');
\i существует такой элемент $\mathbf{0} \in V$, что $\mathbf{x} + \mathbf{0} = \mathbf{x}$ для любого $\mathbf{x} \in V$ (''существование нейтрального элемента относительно сложения''), называемый \it{нулевым вектором} или просто \it{нулём} пространства $V$;
\i для любого $\mathbf{x} \in V$ существует такой элемент $-\mathbf{x} \in V$, что $\mathbf{x} + (-\mathbf{x}) = \mathbf{0}$, называемый вектором, \it{противоположным} вектору $\mathbf{x}$;
\i $\alpha(\beta\mathbf{x}) = (\alpha\beta)\mathbf{x}$ (''ассоциативность умножения на скаляр'');
\i $1\cdot\mathbf{x} = \mathbf{x}$ (''унитарность: умножение на нейтральный (по умножению) элемент поля F сохраняет вектор'').
\i $(\alpha + \beta)\mathbf{x} = \alpha \mathbf{x} + \beta \mathbf{x}$ (''дистрибутивность умножения вектора на скаляр относительно сложения скаляров'');
\i $\alpha(\mathbf{x} + \mathbf{y}) = \alpha \mathbf{x} + \alpha \mathbf{y}$(''дистрибутивность умножения вектора на скаляр относительно сложения векторов'').
\end{enumerate}
\end{solution}

\begin{problem}[31(Lecture\_all.pdf утв. 6.2(2))]
Любое поле $F$ нулевой характеристики содержит $\Q$ в качестве подполя.
\end{problem}

\begin{solution}

\(\tilde{m} := \ubrace{1+\dots+1}{m \text{ штук}} \in F\)

\(\tilde{n} := \ubrace{1+\dots+1}{n \text{ штук}} \in F\)

Для \(m \ne n\) имеем \(\tilde{m} \ne \tilde{n}\) (иначе \(\tilde{m} - \tilde{n} = 0\), и \(\Char F \ne 0\)).

Противоположный к элементу \(\tilde{m}\) обозначим \(-\tilde{m}\).

Получили \(\Z \subset F\).

По №5-6.21a \(\Quot F=F\) для поля (подставим $\Quot F$ вместо $F$).

\bf{Лемма.} (используется для доказательства №5-6.21b) Если \(K \subset L\), то \(\Quot K \subset \Quot L\) для всех колец $K$ и $L$.
\begin{solution}
$K \subset L\Then \Quot K$ вложено в $\Quot L$ как множество. Инъективное отображение строится как $\phi: \Quot K \to \Quot L, \frac{a}{b} \mapsto \frac{a}{b}$. Очевидно, это инъекция (по №3-4.23f).
\end{solution}

У нас \(\Z \subset F \Then \Q=\Quot\Z\subset\Quot F=F\).

\end{solution}

\begin{problem}[32 (Lecture\_all.pdf утв. 6.5(2))]
Пусть $f(x)$ --- неприводимый многочлен степени $n$, и $K = F[x]/(f(x))$. Тогда многочлен $f(x)$ имеет корень в $K$.
\end{problem}

\begin{solution}
Обозначим смежный класс многочлена \(g(x) \in F\) как \(\overline{g}(x) \in K\). Тогда имеем: \(\overline{x} \in K\) --- корень многочлена \(f(x)\), т. к. \(f(\overline{x}) = \overline{f}(\overline{x}) = 0\).
\end{solution}

\begin{problem}[33 (Lecture\_all.pdf утв. 6.5(1))]
Пусть $f(x)$ --- неприводимый многочлен степени $n$, и $K = \sfrac{F[x]}{(f(x))}$. Чему равна степень $[K : F]$ этого расширения?
\end{problem}

\begin{solution}
Обозначим смежный класс многочлена \(g(x) \in F\) как \(\overline{g}(x) \in K\). Рассмотрим \(\ol{1}, \ol{x}, \dots, \ol{x}^{n-1}\). Пусть они ЛЗ, т. е. \(\exists \lambda_0, \lambda_1, \dots, \lambda_{n-1} \in F: \lambda_0\cdot\ol{1}+\lambda_1\cdot\ol{x}+\dots+\lambda_{n-1}\cdot\ol{x}^{n-1} = 0\). Тогда \(g(x) = \lambda_0+\lambda_1x+\dots+\lambda_{n-1}x^{n-1} \in (f(x))\), а по неприводимости \(f(x)\) имеем \(g(x) = 0\), т. е. \(\lambda_0 = \lambda_1 = \dots = \lambda_{n-1} = 0\), и данная ЛК тривиальна. Поэтому \(\ol{1}, \ol{x}, \dots, \ol{x}^{n-1}\) ЛНЗ.

\(\forall\) многочлена \(h(x) \in F[x] \ \ol{h}(x)\) --- образ при факторизации по идеалу \((f(x))\) --- совпадает с \(\ol{r}(x)\), где \(r(x)\) --- остаток от деления \(h(x)\) на \(f(x)\). Поэтому \(\ol{1}, \ol{x}, \dots, \ol{x}^{n-1}\) образуют базис K как линейного пространства над F, т. е. \([K:F] = n\).
\end{solution}

\begin{problem}[34 (7.9,7.10)] Умение находить степень расширения и минимальный многочлен для алгебраического над полем элемента.
\end{problem}

\begin{solution}
Как находить минимальный многочлен $m_\alpha$? Придумать многочлен, у которого $\alpha$ является корнем, и доказать (например, по критерию Эйзенштейна из №24), что он неприводим. Тогда по СЭУ из опр. 29 минимального многочлена, это действительно минимальный многочлен.

По №7.1г(№30г exam\_5-6), степень расширения равна степени минимального многочлена.

\begin{itemize}
\tightlist
\i
  \(\Q(\sqrt2) \supset \Q\): $m = x^2-2$ $\Then$ \(\deg = 2\)
\i
  \(\Q(\sqrt[7]{5}) \supset \Q\): $m = x^7-5$ $\Then$ \(\deg = 7\)
\i
  \(\R(2-3i) \supset \R\): \(\deg = 2\)
  
  $m = 9x^2+4 = (2-3i)(2+3i)$ --- сложно доказывать неприводимость, критерий Эйзенштейна не помогает.
  
  Попробуем воспользоваться теоремой Виета: $\begin{cases} c=(2-3i)(2+3i)=4+9=13 \\ b=(2-3i)+(2+3i)=4 \end{cases}$. $m = x^2-4x+13$. Тоже неудача, критерий Эйзенштейна не помогает.
  
  Замена $x \mapsto x+1$: $m = (x+1)^2-4(x+1)+13 = x^2-2x+10$.
  Применяем критерий Эйзенштейна для $p=2$ и получаем, что $m$ неприводим.
\i
  \(\C(2-3i) \supset \C\): \(\deg = 1\)
  
  $m=x-(2-3i)$
  
  $N(2-3i) = 4+9=13$ --- простое число, значит, $2-3i$ --- простой элемент (по №3.1а(№5-6.9) знаем, что если норма --- простое число, то элемент неразложим, а по №2.8(№5-6.5) в факториальном кольце простота эквивалентна неразложимости (по №7-8.7 евклидово кольцо факториально)).
  
  Применяем критерий Эйзенштейна для $p=2-3i$ и получаем, что $m$ неприводим.
\i
  \(\Q(\sqrt2+\sqrt{3}) \supset \Q\): \(\deg = 4\)
  
  $\alpha := \sqrt2+\sqrt{3} \Then \alpha^2 = 5+2\sqrt6 \Then \alpha^2-5=2\sqrt6 \Then (\alpha^2-5)^2 = 24$
  
  $m = \alpha^4-10\alpha^2+1$ 
  
  Критерий Эйзенштейна не работает. Замена $x \mapsto x+1$: $m = x^4+4x^3-4x^2-16x-8$ --- неприводим по критерию Эйзенштейна при $p=4$.
\i
  \(\Q(1+\sqrt2) \supset \Q(\sqrt2+\sqrt{3})\): \(\deg = 1\)
  
  \bf{Лемма.} $\Q[\sqrt2+\sqrt3] \isom \Q[\sqrt2, \sqrt3]$
  \begin{solution}
	$\subseteq$ очевидно.
	
	$\supseteq$: Рассмотрим $\frac{1}{\sqrt2+\sqrt3} = \frac{\sqrt3-\sqrt2}{3-2} = \sqrt3-\sqrt2$ --- обратный к $\sqrt2+\sqrt3$ (он есть, т. к. $\Q[\sqrt2+\sqrt3] \isom \Q\sqrt2+\sqrt3)$ --- полю.
	
	$(\sqrt2+\sqrt3)+(\sqrt3-\sqrt2) = 2\sqrt3$, т. е. $\sqrt3$ лежит в кольце. Тогда $\sqrt2 = (\sqrt2+\sqrt3)-\sqrt3$ тоже лежит в кольце.
  \end{solution}
  
  Тогда $\Q(\sqrt2+\sqrt{3}) \isom \Q(\sqrt2)(\sqrt{3})$
  
  $m = x-(1+\sqrt2)$ --- степени 1, неразложимый. $1+\sqrt2 \in \Q(\sqrt2)(\sqrt{3})$.
\i
  \(\Q(\omega) \supset \Q\): \(\deg = 2\)
  
  $m = x^2+x+1$: $m(\omega) = 0$ (можно понять по картинке), неприводим по критерию Эйзенштейна после замены $x \mapsto x+1$ для $p=3$, $\deg m = 2$.
\i
  \(\Q(\sqrt[3]{2}, \omega) \supset \Q\): \(\deg = 6\)
  
  $\Q(\sqrt[3]{2}, \omega) \isom \Q[\sqrt[3]{2}][\omega]$ по №7.4а, имея в виду, что $F[\alpha_1, \dots, \alpha_n] \isom F[\alpha_1]\dots[\alpha_n]$ --- очевидно, и что $\sqrt[3]{2}$ и $\omega$ алгебраичны над $\Q$.
  
  $[\Q[\sqrt[3]{2}]:\Q] = 3$, т. к. $m = x^3-2$ --- неприводим по критерию Эйзенштейна.
  
  $\omega \not\in \Q[\sqrt[3]{2}]$, т. к. в $\Q[\sqrt[3]{2}]$ нет комплексных чисел. Значит, $m = x^2+x+1$ неприводим над $\Q[\sqrt[3]{2}]$ по критерию Эйзенштейна.
  
  Итого, пользуясь №7.3(№32 exam\_5-6), $3\cdot 2 = 6$
\i
  \(\Q(\sqrt[4]{2}, i) \supset \Q\): \(\deg = 8\)
  
  Аналогично, $4\cdot2 = 8$.
\i
  \(\Q(\sqrt[5]{2}, i) \supset \Q\): \(\deg = 10\)
  
  Аналогично, $5\cdot2 = 10$.
\end{itemize}

\bf{TODO: проставить ссылки на утверждения}
\end{solution}

\begin{problem}[35(6.10,8.9а,10.5)] Умение описывать расширения степени 2: минимальный многочлен, поле разложения, нормальность, группа Галуа.
\end{problem}

\begin{solution}
На примере \(\Q(\sqrt{2}) \supset \Q\).
Степень 2. Мин. многочлен \(x^2-2\) --- степени 2.
\(\Q(\sqrt2)\) --- поле разложения многочлена \(x^2-2\), т. к. все его корни \(\sqrt2, -\sqrt2 \in \Q(\sqrt2)\).
Не существует промежуточных, потому что тогда они имеют степень 1 или 2, т. е. совпадают с \(\Q(\sqrt2)\) или с \(\Q\).

Рассмотрим автоморфизмы \(\Q(\sqrt2) \to \Q(\sqrt2)\). Их всего 2:

\(a+\sqrt2b \mapsto a+\sqrt2b\)

\(a+\sqrt2b \mapsto a-\sqrt2b\)

Других нет, т. к. достаточно рассматривать только перестановки корней минимального многочлена:

\(\sqrt2 \mapsto \pm \sqrt2\)

\(-\sqrt2 \mapsto \mp \sqrt2\)

Другие не рассматриваются, поскольку 1 и ЛНЗ корни многочлена образуют базис в \(\Q(\sqrt2)\) как линейном пространстве над \(\Q\); \(1\) при этом должен перейти в 1, т. к. \(\Q\) сохраняется.

Таких автоморфизмов 2 \(\Then\) степень расширения 2 \(\Then\) это расширение Галуа. Группа из двух элементов --- \(\Z_2\) --- то группа Галуа данного расширения (других (с точностью до изоморфизма) групп из 2-х элементов не бывает).
\end{solution}

\begin{problem}[36(9.1)]
Для производной выполнены формулы $(f+g)'=f'+g'$ и $(fg)' = f'g+fg'$.
\end{problem}

\begin{solution}
Для \(f(x) = a_nx^n+\dots+a_1x+a_0\) и \(b(x) = b_nx^n+\dots+b_1x+b_0\):

\begin{itemize}
\i \((f+g)'= n(a_n+b_n)x^{n-1}+\dots+(a_2+b_2)x+(a_1+b_1) = (na_nx^{n-1}+a_2x+a_1)+(nb_nx^{n-1}+\dots+b_2x+b_1) = f'+g'\)

\i Рассмотрим \(f(x)-f(y) = \sum\limits_{k=1}^{n} a_k(x^k-y^k) = (x-y)\sum\limits_{k=1}^{n} a_k(x^{k-1}+x^{k-2}y+\dots+y^{k-1}) = (x-y) \Phi(x, y)\), где \(\Phi(x, y) = \sum\limits_{k=1}^{n} a_k(x^{k-1}+x^{k-2}y+\dots+y^{k-1})\). Заметим, что \(\Phi(x, x) = f'(x)\).
	Тогда имеем для \(\phi = fg\): \(\phi(x)-\phi(y) = f(x)g(x)-f(y)g(y) = f(x) (g(x)-g(y))+g(y)(f(x)-f(y)) = (x-y)[f(x)G(x,y)+g(y)\Phi(x,y)]\).
	Отсюда \(\phi' = f(x)G(x,x)+g(x)\Phi(x,x)=f(x)g'(x)+g(x)f'(x)\).
\end{itemize}
\end{solution}

\begin{problem}[37 (9.2)]
(для $\Char F = 0$) Многочлен $f$ не имеет кратных корней тогда и только тогда, когда $(f,f')=1$.
\end{problem}

\begin{solution}
Пусть \(f(x) = (x-a)^mf_1(x), f_1(x) \not\Divby (x-a), m \ge 2\). Тогда \(f'(x)=m(x-a)^{m-1}f_1(x)+(x-a)^mf'_1(x)\).

\begin{itemize}
\tightlist
\i
  Если \(m>1\), то \(f'(a) = 0\).
\i
  Если \(m=1\), то \(f'(x)=(x-a)f'_1(x)+f_1(x) \Then f'(a)=f_1(a) \ne 0 \Then f(x)\) имеет кратные корни \(\Iff\) эти корни являются корнями \(f'(x)\).
\end{itemize}

\end{solution}

\begin{problem}[38(9.6)]
Докажите, что можно построить\\ 
a) все точки с рациональными координатами;
b) $\xi_n$, где $n=3,4,6$;

Если мы построили точки $z,w$, то можно ли построить точки 
c) $\overline{z},-z$? 
d) $z+w$, $z-w$?
e) $z\cdot w$?
\end{problem}

\begin{solution}
При помощи гугла учимся строить: перпендикулярную прямую (через заданную точку), параллельную прямую (через заданную точку).

Как обосновать то, что мы можем брать раствор циркуля, равный расстоянию между какими-то двумя точками, и переносить его на другое место?
Пусть есть отрезок \(AB\), и мы хотим окружность с радиусом \(AB\) и центром в т. \(C\).

\includegraphics[width=1.56250in]{parallelogram.png}
Строим параллелограмм как на рисунке.
\(CD = AB\).

\begin{enumerate}
\def\labelenumi{\alph{enumi})}
\i
  Берём отрезок \(01\) и произвольную точку \(A\), не лежащую на нём. Проводим \(0A\). На луче \(0A\) начиная от точки \(0\) откладываем \(n\) равных отрезков произвольной длины. Пусть их концы, лежащие на \(0A\), есть \(A_1, \dots, A_n\) (считая от точки \(0\)). Проводим \(A_n1 =: A_n B_n\) и параллельно ей \(A_{n-1}B_{n-1}, \dots, A_1B_1\). По теореме Фалеса \(0B_1 = B_1B_2 = \dots = B_{n-1}1\). Сделав так для любого \(n\), получим все точки с рациональными координатами на \(01\), размножить на ось OX тривиально, получить так же поделенную ось OY тривиально, а т. к. любая точка однозначно задаётся проекциями и мы умеем строить перпендикуляры, можем строить любую точку с рациональными координатами.
\i
  Шестиугольник --- откладывая на окружности хорды длиной с радиус, треугольник --- по шестиугольнику, четырёхугольник --- строя перпендикуляр из центра окружности, в которую он вписан.
\i
  Отражение относительно осей.
\i
  Тривиально.
\i
  В экспоненциальной записи: \(z\omega = r_1e^{i\phi_1}r_2e^{i\phi_2}=(r_1r_2)e^{i(\phi_1 +\phi_2}\).
  Итого, надо научиться строить сумму углов и отрезок с длиной, равной произведению двух других. Сумма углов: тривиально.
  Произведение: пользуемся теоремой из геометрии о соотношении высоты прямоугольного треугольника со всякими другими отрезками (та, которая выводится из подобия).

  \includegraphics[width=2.08333in]{triangle.png}
  Взяв \(BH=h^2, AH=a^2, CH=b^2\), получим \(BH^2=AH\cdot CH \Then BH=ab\).
  Как строить \(a^2\) и \(b^2\)?
  Взяв \(BH=a^2, AH=1, CH=x\), получим \(a^2=1\cdot x \Then x=a^2\).
\end{enumerate}

\end{solution}

\begin{problem}[39 (9.12а)]
Докажите невозможность \bf{удвоения куба}, то есть построение куба объёма $2$, имея куб объёма $1$ с помощью циркуля и линейки.
\end{problem}

\begin{solution}
Задача сводится к построению циркулем и линейкой числа \(\sqrt[3]{2}\).

Минимальный многочлен для $\sqrt[3]{2}$ есть $x^3 - 2$, неприводимый по признаку Эйзенштейна в $\Q$ и потому минимальный.
Значит, размерность расширения $= 3$ $\Then$ не существует башни промежуточных расширений размерности $\le 2$ $\Then$ по №19 exam\_7-8 её нельзя построить.

\end{solution}

\begin{problem}[40 (10.2)]
Пусть $\varphi: F \to F$ --- \it{автоморфизм поля} $F$ (изоморфизм поля на себя). 
a) Пусть $\mathrm{char} F =0$. Верно ли, что $\varphi$ сохраняет $\Q$? (то есть при $q\in\Q$ выполнено равенство $\varphi(q)=q$).
b) Пусть $\mathrm{char} F =p$. Верно ли, что $\varphi$ сохраняет $\Z_p$?
\end{problem}

\begin{solution}
\begin{enumerate}
\def\labelenumi{\alph{enumi})}
\i Автоморфизм переводит единицу в единицу: \(\phi(1) = 1\) по свойствам гомоморфизма.
Тогда \(\forall p \in \Z \Have \phi(p) = \phi(\ubrace{1+\dots+1}{p} = \ubrace{1+\dots+1}{p} = p\). Получили, что \(\Z\) сохраняется.

Тогда $\phi(2) = \phi(1+1) = \phi(1)+\phi(1)=1+1=2$.
И так далее. Получили, что $\Z$ сохраняется.

Если какой-нибудь элемент $\frac{a}{b} \in \Q$ перевёлся не в себя, то $\phi(\ubrace{\frac{a}{b}+\dots+\frac{a}{b}}{b}) = \ubrace{\phi(\frac{a}{b})+\dots+\phi(\frac{a}{b})}{b} \ne a$, т. е. в $\Z$ что-то перешло не в себя. Противоречие.
%Имеем: \(\frac{p}{q} \ubrace{\~}{\text{определение экв-ти}} \frac{pq^{-1}}{1} \ubrace{\~}{\text{изоморфизм}} pq^{-1}\)
%
%Тогда \(\phi(\frac{p}{a}) = \phi(pa^{-1}) = \phi(p)\ubrace{[\phi(a)]^{-1}}{\text{свойство гомоморфизма}} = pa^{-1} = \frac{p}{a}\)

\i Да, т. к. если \(m \in \Zp\), то \(\phi(m) = \ubrace{\phi(1)+\dots+\phi(1)}{m} = m\)
\end{enumerate}
\end{solution}

\begin{problem}[41 (10.4) (Lectures\_all.pdf задача 9.1, утв. 9.1)]
Пусть $F \subset K$ — расширение полей. Множество автоморфизмов $K$, оставляющих $F$ на месте, является группой и называется группой автоморфизмов и обозначается $\Aut_F(K) = \Aut([K : F])$.
a) $\Aut_F(K)$ — группа.
b) Пусть $H \subset \Aut_F(K)$ --- подгруппа. Тогда $K^H = \{x \in K \mid \forall h \in H \Have h(x) = x\}$ является полем, причём $K \supset K^H \supset F$.
\end{problem}

\begin{solution}
\begin{enumerate}
\def\labelenumi{\alph{enumi})}
\i Композиция автоморфизмов, сохраняющих \(F\), --- автоморфизм, сохраняющий \(F\) \(\Then\) замкнутость.

\(id\) --- нейтральный элемент.

Ассоциативность следует из свойств композиции.

Обратный существует, т. к. автоморфизм --- биекция. Обратный сохраняет \(F\).

\i Пусть \(a, b \in K^H, h \in H\). Тогда \(h(a + b) = h(a) + h(b) = a + b\), и поэтому \(a + b \in K^H\). Аналогично, \(ab \in K^H\). С другой стороны, \(h \in H \subset G\), и поэтому \(h\) сохраняет \(F\). Значит, \(F \subset K^H\).
\(K^H \subset K\) по определению.
\end{enumerate}
\end{solution}

\begin{problem}[42 (10.5)]
Опишите группы автоморфизмов $\Q(\sqrt[3]{2})$.
\end{problem}

\begin{solution}
По №40 знаем, что автоморфизм сохраняет $\Q$. Значит, нужно смотреть только за тем, куда переходит $\sqrt[3]{2}$.
Рассмотрим минимальный многочлен для $\sqrt[3]{2}$: $m = x^3-2$.
Куда может перейти $\sqrt[3]{2}$? В $\sqrt[3]{2}, \ubrace{\sqrt[3]{2}\omega, \sqrt[3]{2}\omega^2}{\not\in \Q(\sqrt[3]{2})}$. Значит, он может перейти только в себя. \(\Aut_\Q \Q(\sqrt[3]{2}) = \{id\}\), т. к. все другие перестановки корней дают комплексные числа.

Знаем, что любой автоморфизм задаётся перестановкой корней минимального многочлена.

Пусть \(\phi\) --- авторморфизм. Тогда \(\phi(m_\gamma(\gamma)) = 0 \Then 0 = a_n \phi(\gamma)^m+\dots+a_0 \Then \phi(\gamma)\) --- корень \(m_\gamma\) \(\Then\) сопряжён с \(\gamma\).

Степень расширения равна степени минимального многочлена, т. е. $3$.
\end{solution}

\begin{problem}[43 (11.1) (Lectures\_all.pdf теор. 11.1)]
\begin{enumerate}
\def\labelenumi{\alph{enumi})}
\i Конечное поле характеристики $p$ состоит из $p^n$ элементов. 
\i Поле $F$ является полем разложения многочлена $x^{p^n}-x$. 
\i Cуществует единственное поле из $p^n$ элементов.
\end{enumerate}
\end{problem}

\begin{solution}
\begin{enumerate}
\def\labelenumi{\alph{enumi})}
\i Так как \(K\) --- конечное расширение поля \(\Zp\), то \(K\) является n-мерным линейным пространством над \(\Zp\), и поэтому состоит из \(p^n\) элементов.
\i Пусть \(\alpha \in K, \alpha \ne 0\). Тогда \(\alpha^{p^n-1} = 1\). Следовательно, \(\alpha\) является корнем многочлена \(f\). Степень многочлена \(f\) равна \(p^n\), все элементы K являются его корнями. Ясно, что \(K\) --- минимальное поле, в котором \(f\) раскладывается на линейные множители. Следовательно, K --- его поле разложения.
\i Поле разложение многочлена \(f\) единственно с точностью до изоморфизма.
\end{enumerate}
\end{solution}

\begin{problem}[44 (11.2)]
Найдите все неприводимые многочлены (со стар. коэффициент $1$) степени $2$, $3$ над полем a) $\F_2$, b) $\F_3$.
\end{problem}

\begin{solution}
Выписываем все возможные многочлены и вычёркиваем те, которые разложимы. Если мы вычеркнули \it{все} разложимые (а мы умеем это делать, см. пункт а), то остались только неразложимые. Чтобы понять, что мы ничего не пропустили, проверяем оставшиеся на неразложимость при помощи признака Эйзенштейна, не забывая, что можно делать замену переменных. Примеры того, как это происходит, можно увидеть в №34.

\bf{Зам.} Автор не проводил подобную проверку.

\begin{enumerate}
\def\labelenumi{\alph{enumi})}
\i
  \(\F_2\):
  Неразложимые степени 1: x и x+1. Разложимые степени 2 --- какая-то комбинация многочленов степени 1. Оставшиеся --- неразложимы. Теперь смотрим всё возможные многочлены степени 3, которые можно получить, перемножая многочлены степени 1 и многочлены степени 2.

  \(\deg=2\): \(\s{x^2}, \ubrace{\s{x^2+1}}{(x+1)(x+1)}, \s{x^2+x}, x^2+x+1\)

  \(\deg=3\): \(\s{x^3}, \ubrace{\s{x^3+1}}{(x^2+1)(x^2+x+1)}, \s{x^3+x}, x^3+x+1, \s{x^3+x^2}, x^3+x^2+1, \s{x^3+x^2+x}, \ubrace{\s{x^3+x^2+x+1}}{(x^2+1)(x+1)}\)
\i
  \(\F_3\):
  Рассуждения аналогичны.

  \(\deg=2\): \(\s{x^2}, x^2+1, \ubrace{\s{x^2+2}}{(x+2)(x+1)}, \s{x^2+x}, x^2+x+1, \ubrace{\s{x^2+x+2}}{(x+1)(x+2)}, \s{x^2+2x}, x^2+2x+1, x^2+2x+2\)

  \(\deg=3\): \(\s{x^3}, \ubrace{\s{x^3+1}}{(x^2+1)(x^2+x+1)}, x^3+2, \s{x^3+x}, x^3+x+1, x^3+x+2, \s{x^3+2x}, x^3+2x+1, x^3+2x+2, \s{x^3+x^2}, x^3+x^2+1, x^3+x^2+2, \s{x^3+x^2+x}, \ubrace{\s{x^3+x^2+x+1}}{(x^2+1)(x+1)}, x^3+x^2+x+2, \s{x^3+x^2+2x}, x^3+x^2+2x+1, \ubrace{\s{x^3+x^2+2x+2}}{(x^2+2)(x+1)}, \s{x^3+2x^2}, x^3+2x^2+1, x^3+2x^2+2, \s{x^3+2x^2+x}, x^3+2x^2+x+1, \ubrace{\s{x^3+2x^2+x+2}}{(x^2+1)(x+2)}, \s{x^3+2x^2+2x}, \ubrace{\s{x^3+2x^2+2x+1}}{(x^2+2)(x+2)}, x^3+2x^2+2x+2\)
\end{enumerate}

\end{solution}

\begin{problem}[45 (11.3)]
Постройте поле из a) $4$; b) $8$; c) $9$ элементов.
\end{problem}

\begin{solution}

Для \(p^n\): \(F_{p^n} = \sfrac{F_p[x]}{(f(x))}\), где \(f(x)\) --- неприводимый многочлен степени n (пользуемся №33: $[\sfrac{F[x]}{(f(x))}:F] = n$). Итого, надо просто найти неприводимый многочлен над \(F_p\) степени \(n\). Как искать неприводимые многочлены рассказано в №44.

\begin{enumerate}
\def\labelenumi{\alph{enumi})}
\tightlist
\i
  \(4=2^2\)
  \(\F_4 = \sfrac{F_2[x]}{(x^2+x+1)} = \{0; 1;x;1+x\}\)
\i
  \(8=2^3\)
  \(\F_8 = \sfrac{F_2[x]}{(x^3+x^2+1)} = \{0; 1; x; x+1; x^2; x^2+1; x^2+x; x^2+x+1 \}\)
\i
  \(9=3^2\)
  \(\F_9 = \sfrac{F_3[x]}{(x^2+1)}\)
\end{enumerate}

Чтобы выписать элементы кольца явно, берём все возможные многочлены нужной степени над нужным полем и делим с остатком на многочлен, по которому факторизуем (факторизация в данном случае и есть деление с остатком).

\end{solution}

\end{document}
