\input{boilerplate.tex}

\date{}

\begin{document}

\begin{problem}[1 (1.4) [Каргальцев]]
Для любого числа $u \in \C$ определим множество $\Z[u] = \cup_{n = 0}^{\infty} \{a_0 + a_1u + \ldots + a_nu^n | a_0, a_1, \ldots, a_n \in \Z\}$.
\end{problem}

\begin{solution}
а) Докажите, что \(\Z[u]\) является областью целостности.

То, что \(\Z[u]\) кольцо проверяется непосредственно. Поскольку \(\Z[u] \subset \C\) и \(\C\) --- область целостности (\it{потому что $\C$ --- поле}), то и \(\Z[u]\) область целостности.

б) При каких \(u \in \C\) данное \(\Z[u]\) ``конечномерно над \(\Z\)'', то есть найдётся такое \(N\), что \(\Z[u] = \cup_{n = 0}^{\infty} \{a_0 + a_1u + \ldots + a_nu^N | a_0, a_1, \ldots, a_N \in \Z\}\)?

Покажем, что \(\Z[u]\) ``конечномерно над \(\Z\)'', \(\Iff \exists f \in \Z[x]: f(u) = 0\), \(f \ne 0\) и старший коэффициент \(f(x)\) равен \(1\) \(\,\,\,(*)\).

\(\Then\)

Поскольку \(u^{N + 1} \in \Z[u] \Then \exists a_0, \ldots, a_N \in \Z: u^{N + 1} = \sum\limits_{0}^{N} a_ku^k \Then u\) --- корень \(f(x) = x^{N + 1} - \sum\limits_{0}^{N} a_kx^k\)

\(\When\)

Пусть \(u\) --- корень многочлена \(f(x) = u^{N} + \sum\limits_{0}^N a_kx^k\), удовл. условию \((*)\). Тогда \(u^N\) выражается через меньшие степени. (\(u^N = -\sum\limits_{0}^{N - 1}a_ku^k\))

Индукцией по \(k \geqslant N\) легко показать, что \(u^k\) выражается через \(1, u, \ldots u^{N - 1}\).

(\(u^{k + 1} = u \cdot u^{k} \stackrel{\textup{предположение индукции}}{=} u \cdot (\sum\limits_0^{N - 1} b_ku^k) = (\sum\limits_1^{N - 1} b_{k - 1} u^k) + b_{N - 1}u^N \stackrel{\textup{база индукции}}{=} (\sum\limits_1^{N - 1} b_{k - 1} u^k) + b_{N - 1}\sum\limits_{0}^{N - 1}-a_ku^k\)

\end{solution}

\begin{problem}[2. (1.2 )]
Для комплексного числа $z \in C$ введём норму $N(z) = |z|^2$.

а) $N(zw) = N(z)N(w)$.

Для каждого $z \in D$:

б) Верно ли, что $N(z)$ — натуральное число?

в) Верно ли, что $N(z) = 1 \Leftrightarrow z$ — обратим?
\end{problem}

\begin{solution}

а) Просто проверим: \(N(zw)= N(a_z + b_zi)(a_w + b_wi)) = N(a_za_w - b_zb_w + (a_zb_w + a_wb_z)i)=\)
\((a_za_w-b_zb_w)^2 + (a_zb_w + b_za_w)^2 =\)
раскрыли скобки
\(= (a_z^2 + b_z^2)(a_w^2 + b_w^2) = N(z)N(w)\)

б) Заметим, что \(\Z[i] = \{a + bi\,|\,a,b \in \Z\}\)

Значит, \(|a + bi| = a^2 + b^2 \in |N\). Аналогично:

\(\Z[2i] = \{a + 2bi\,|\,a,b \in \Z\} \Rightarrow |a + 2bi| = a^2 + 4b^2 \in \N\)

\(\Z[\sqrt{2}i] = \{a + \sqrt{2}bi\,|\,a,b \in \Z\} \Rightarrow |a + \sqrt{2}bi| = a^2 + 2b^2 \in \N\)

\(\Z[\sqrt{3}i] = \{a + \sqrt{3}bi\,|\,a,b \in \Z\} \Rightarrow |a + \sqrt{3}bi| = a^2 + 3b^2 \in \N\)

в) \(\Then\)

\(N(z) = a^2 + b^2 = 1\)

\(\frac{1}{z} = \frac{1}{a+bi} = \frac{a-bi}{a^2 + b^2} = \frac{a-bi}{1} = a-bi = \bar{z}\),
а \(z\) и \(\bar{z}\) одновременно лежат в \(D\), значит \(\exists z^{-1} = \bar{z}\).

\(\When\)

\(zz^{-1} = 1 \Rightarrow \begin{cases} N(zz^{-1}) = N(z)N(z^{-1}) = 1 \\ N(z) = a^2 + b^2 \geqslant 1  \end{cases} \Rightarrow N(z) = 1\)

\end{solution}

\begin{problem}[3]
Пример нефакториального кольца вида $Z[u]$.
\end{problem}

\begin{solution}
Пример: \(\Z[2i]\) не является факториальным кольцом, потому что \(4 = 2\cdot2 = (2i)(-2i)\),
но при этом \(2 \nsim 2i\) -- противоречие с единственностью разложения в факториальном кольце.

Еще пример: \(\Z[\sqrt{3}i]\) (аналогичное рассуждение \(4 = 2\cdot 2 = (1+\sqrt{3}i)(1-\sqrt{3}i)\)).
\end{solution}

\begin{problem}[4 (2.7) [Каргальцев]]
Простой элемент области целостности является неразложимым.
\end{problem}

\begin{solution}
Пусть \(p\) --- простой и \(p = xy \Then x | p \land y | p\) . Из определения
простоты \(p | x \lor p | y\). Но тогда или \(x | p \land p | x\), или
\(y | p \land p | y\). Тогда \(p \sim y \lor p \sim x \Then y \in K^* \lor x \in K^*\),
то есть \(p\) --- неразложимый.
\end{solution}

\begin{problem}[5 (2.8)]
В факториальном кольце любой неразложимый элемент является простым.
\end{problem}

\begin{solution}
Пусть \(x=ab\) -- неразложимый. \(x = ab \Then x\,|\,ab\).

\(x\) неразложимый, значит б.о.о. \(a\in K^*\). Тогда в силу единственности разложения
\(x = ab = ap_1\ldots p_k \Then x \sim b \Then x\,|\,b\).
\end{solution}

\begin{problem}[6 (часть 2.9) [Каргальцев]]
$K$ --- евклидово кольцо. Верно ли, что если для $a, b \ne 0$ выполнено равенство $N(ab) = N(a)$, то $b$ обратим?
\end{problem}

\begin{solution}

Поделим \(a\) с остатком на \(ab\):

\[a = abq + r: r=0 \lor N(r) < N(ab)\].
\[r = a(1 - bq)\].

Если \(r=0\), то \(bq = 1\) и \(b\) обратим. Иначе \(N(ab) > N(r) = N(a(1 - bq)) \geqslant N(a) = N(ab)\). Противоречие.
\end{solution}

\begin{problem}[7 (2.10)]
Геометрический способ доказательства того, что $\Z[i]$, $\Z[\omega]$ — евклидово кольцо.
\end{problem}

\begin{solution}
ВСТАВИТЬ КАРТИНКУ
Пусть \(a, b \in \Z[i]\). Поделим \(a\) на \(b\) с остатком:

\(a = pb + q\).

Надо доказать, что если \(q \neq 0\), то \(N(q) < N(b)\). Рассмотрим точку \(\frac{a}{b}\),
пусть ближайший к ней узел в решетке \(p\), тогда \(\frac{a}{b} = p + \frac{q}{b}\).
Но \(\frac{q}{b}\) по модулю меньше половины диагонали единичного квадрата
\(\left|\frac{q}{b}\right| \leqslant \left|\frac{\sqrt{2}}{2}\right| \leqslant 1\),
т.е. \(|q|^2 < |b|^2 \Then N(q) < N(b)\), если \(\frac{q}{b}\) не совпадает с центром квадрата.

(TODO иначе)

\(\Z[\omega]\) аналогично.
\end{solution}

% -----------------------------------------------------------

\begin{problem}[9 (3.1)]
а)~Если $p$ -- простое целое число и существует такое $z \in D$, что 
$N(z) = p$, то $z$ — неразложимый элемент.

б)~Если $p$ -- простое целое число и не существует такого $z \in D$, 
что $N(z) = p$, то $p$ -- неразложимый элемент.

в)~Если $D$ -- факториальное кольцо, то для любого неразложимого элемента
$z \in D$ либо $N(z) = p$, либо $z \sim p$ для некоторого целого простого числа $p$.

\end{problem}

\begin{solution}
а)~Имеем: $z \in D$, $N(z) = p$. Пусть $z=bc$, тогда $N(z) = N(b)N(c) = p \Then$ $N(b) = 1$ или $N(c) = 1$, т.е $b \in D^*$ или $с \in D^* \Then$ $z$ -- неразложим (исп. задачу 2в)

б)~Пусть $p = bc$. Тогда $N(p) = N(b)N(c) = p^2$. Два случая:

\begin{itemize}
	\item $N(b) = N(c) = p$ --- невозможно по условию
	\item $N(a) = 1,\,\,N(b) = p^2$ или  $N(a) = p^2,\,\,N(b) = 1 \Then$ $b \in D^*$ или $c \in D^*$ (исп. задачу 2в).
\end{itemize}

в)~$N(z) = z\bar{z} = p_1^{k_1} \cdot \ldots \cdot p_m^{k_m}$ (в силу факториальности кольца). $z$ неразложим $\Then \exists i\colon p_i \vdots z \Then zk=p_i$

$N(p_i) = p_i^2=N(z)N(k)\Then$ либо $N(z)=p_i\Then$ $z$ -- неразложим, либо $N(z) = p_i^2\Then z \sim p_i$.
 
\end{solution}

% --------------------------------------------------------------

\begin{problem}[10 (2.7) [Каргальцев]]
	Если $z \in D$, $z | x$, и $N(z) = N(x)$, то $z \sim x$.
\end{problem}

\begin{solution}
	Пусть \(x = yz\). Тогда \(N(yz) = N(z) \Then y\) обратим (по №6) и, значит, \(x \sim z\).
\end{solution}

% ------------------------------------------------------------------

\begin{problem}[11 (3.3) [Каргальцев]]
	(Простые гауссовы числа) Пусть $p$ --- простое целое число.
\end{problem}

\begin{solution}
	
	а) Если \(p\) = \(4k + 3\), то \(p\) --- неразложим в \(\Z[i]\).
	
	Если \(p\) разложим, тогда \(p = z\bar{z} = Re^2z + Im^2z\). Но число, дающее остаток 3 при делении на 4 не быть представлено в виде суммы двух квадратов (квадраты дают остаток 1 при делении на 4).
	
	б) Если \(p = 4k + 1\), то \(p\) --- разложим в \(\Z[i]\).
	
	Если \(p = 4k + 1\), то \(-1\) --- вычет по модулю \(p\), т. е \(\exists x \in \Z: p| x^2 + 1 \Then p | (x + i)(x - i)\). Если \(p\) --- неразложим, тогда \(p\) --- прост и либо \(p| (x + i)\), либо \(p | (x - i)\). 
	
	\begin{itemize}
		\item $p| (x + i)\Then x + i = p(c + di)\Then 1 = pd \Then p\,|\,1$ -- плохо.
		\item $p| (x - i)\Then x - i = p(c + di)\Then -1 = pd \Then p\,|\,1$ -- плохо.
	\end{itemize}
	Значит, \(p\) разложим.
	
	в) Если \(p = 4k + 1\), то \(p = z\bar{z}\), где \(z\) --- неразложим в \(\Z[i]\).
	
	Следует из предыдущего пункта и пункта г) предыдущей задачи.
	
	г) Неразложимые элементы \(\Z[i]\), не описанные в предыдущих пунктах --- \(\pm 1 \pm i\).
	
	Неразложимые элементы, не описанные в предыдущих задачах могут иметь норму или 2, или 4. Норму 4 имеет только \(2\) и ассоциированные с ней, но \(2 = (1 + i)(1 - i)\).
	
	С другой стороны, \(N(\pm 1 \pm i) = 2\), то есть силу пункта в) предыдущей задачи \(\pm 1 \pm i\) неразложимы.
\end{solution}

% ------------------------------------------------------------

\begin{problem}[12 (3.10)]
Евклидово кольцо является кольцом главных идеалов.
\end{problem}

\begin{solution}
Пусть $K$ -- евклидово кольцо, $a \in K$, причем \[N(a) = \min_{x\in K \setminus \{0\}}N(x)\].

Предположим, что $K \neq (a) \Then \exists b\in K\setminus (a) \Then b = aq + r$, где либо $r = 0$, либо $N(r) < N(a)$.

\begin{itemize}
	\item $r=0\Then b=aq\Then b \in (a)$ -- противоречие
	\item $N(r) < N(a) \Then r = b-aq \in I$ -- противоречие с минимальностью нормы $a$.
\end{itemize}
\end{solution}

% ------------------------------------------------------------

\begin{problem}[13 (3.13)]
	
Пусть $D = \Z[i]$ или $\Z[\omega]$. 

а) Верно ли, что из $a | b$ следует, что $N (a) | N (b)$? 

б) Верно ли, что из $\textup{НОД}(N (a), N (b)) = 1$, следует $\textup{НОД}(a, b) = 1$? 

в) Пусть $\textup{НОД}(N (a), N (b)) = p$ -- простое целое число, причём
$p \not| a$, $p \not| b$. Тогда $p$ -- разложим, и если $p = z\bar{z}$, то либо $z$ и $\bar{z}$ порождает идеал $(a, b)$, либо $z$ делит одно
из этих чисел, а $\bar{z}$ — другое.
	
\end{problem}
\begin{solution}
а)~Верно. $a\,|\,b \Then b=ak \Then N(b) = N(a)N(k)$ (по свойству нормы в $D$) $\Then N(a)\,|\,N(b)$

б)~$\textup{НОД}(N (a), N (b)) = p$

Допустим, что $\textup{НОД}(a, b) = k \notin D^*$. 

Тогда $a = kx, b = ky$, и
\begin{equation}
\left. \begin{gathered}
N(a) = N(kx) = N(k)N(x) \\
N(b) = N(ky) = N(k)N(y)
\end{gathered} \right \} \Then \textup{НОД}(N (a), N (b)) \ne 1
\end{equation} -- противоречие. Значит, $\textup{НОД}(a, b) = 1$.

в)~$\textup{НОД}(N (a), N(b)) = p$

Покажем, что $p$ -- разложим. $N(a) = a\bar{a} = pt, p \not| \,a$, $p$ -- простое число $\Then$  допустим, что $p$ неразложима: $p\,|\, \bar{a}$ (по свойству факториального кольца). Тогда $\bar{a} = x - iy\,\vdots\,p \Leftrightarrow x \,\vdots\,p, y \,\vdots\,p \Then a \,\vdots\,p$ -- противоречие $\Then p$ -- разложим.

\begin{equation}
\left\{
\begin{gathered} 
a\bar{a}  = z\bar{z}k \\
b\bar{b} = z \bar{z}l
\end{gathered}
\right. \Then
\left[
\begin{gathered} 
a\,\vdots\,z \textup{ и } b\,\vdots\,\bar{z} \\
a\,\vdots\,\bar{z} \textup{ и } b\,\vdots\,z \\
a\,\vdots\,z \textup{ и } b\,\vdots\,z
\end{gathered}
\right.
\end{equation} 

В последнем случае идеал $(t) = (a,b) \subseteq (z, \bar{z})$ -- очевидно.
Докажем в обратную сторону, что $(z, \bar{z})\subseteq (a,b)$

$z\bar{z} = a\bar{a}\xi + b\bar{b}\eta \,\vdots\,t$

\end{solution}


% ------------------------------------------------------------

\begin{problem}[14 (3.14)]
Умение находить порождающий элемент идеала в кольце $\Z[i]$.
\end{problem}
\begin{solution}
Возможный вариант решения: найдем нормы двух чисел, потом найдем $n$ -- НОД этих норм. После этого переберем все числа, которые имеют норму $n$ и проверим их на то, что они являются порождающим элементом. При этом искать можно только в первой четверти комплексной плоскости (т.к. найдя одно число, получаем сразу 4 поворотами на $\pi / 2$). Если не один из них не подойдет, то проделаем то же самое со всеми делителями $n$ в порядке уменьшения модуля, пока не дойдем до $1$.

Рассмотрим пример:

3.14. Найти порождающий элемент $(11+7i, 18-i)$ в $\Z[i]$.
\textit{Решение.} Заметим, что $\Z[i]$ -- евклидово кольцо, значит, оно является КГИ $\Then$ все идеалы главные $\Then$ идеал $(11+7i, 18-i)$ порождается одним элементом $(t)$. Найдем этот элемент.

$N(11+7i) = 170$

$N(18-i) = 325$

$\textup{НОД}(170, 325) = 5$.
	
Перебором выясняем, что в первой четверти числу с  нормой $5$ соответствуют два числа: $1 + 2i$ и $2 + i$.

Заметим, что $1 + 2i$ не может быть порождающим элементом:

$\frac{18-i}{1+2i} = \frac{(18-i)(1-2i)}{(1+2i)(1-2i)} = \ldots = \frac{16}{5} - \frac{37}{5}i \notin \Z[i]$

С $2+i$ тоже плохо:

$\frac{11+7i}{2+i} = \frac{29}{5} + \frac{3}{5}i \notin \Z[i]$.

Следовательно, среди чисел с нормой $5$ нет $\textup{НОД}(a, b) \Then $ его норма $1 \Then (11+7i, 18-i) = (1)$.   
\end{solution}

% ------------------------------------------------------------

\begin{problem}[15]
	Пусть $I \subset K$ является подмножеством, для которого выполнено следующее условие: для любых $a \in K$, $x \in I$, $y \in I$ верно, что $x + y \in I$, $ax \in I$. Верно ли что это условие равносильно тому, что I --- идеал?
\end{problem}

\begin{solution}

\(\Then:\)

В предположении, что $I$ непусто:

\begin{enumerate}

\item $(I, +) \subset (K, +)$ --- подгруппа по сложению. 
	
	\begin{itemize}
	
	\item Замкнутость по сложению --- дана по условию.
	
	\item Нейтральный по сложению лежит в $I$: действительно, возьмем произвольный $x \in I$ и $a = 0 \in K$: тогда $0 = ax \in I$.
	
	\item Обратный по сложению лежит в $I$: т.к. $-1 \in K$, то $\forall x \in I: -x = (-1) \cdot x \in I$.
				
	\end{itemize}

\item $\forall a \in K, x \in I: ax \in I$ --- дано по условию.

\end{enumerate}

\(\When:\)

\begin{enumerate}
	
	\item $\forall x \in I, y \in I: x + y \in I$ --- выполнено, т.к. идеал --- подгруппа по сложению.
	
	\item $\forall a \in K, x \in I: ax \in I$ --- выполнено по определению идеала.
	
\end{enumerate}


\end{solution}

% ------------------------------------------------------------

\begin{problem}[16 (3.17)]

а) Идеал $(x, y)$ кольца $\Q[x, y]$ конечно порождён, но не является главным.

б) Приведите пример области целостности $K$ и идеала $I$, который не конечно порождён.

\end{problem}

\begin{solution}

а) $(x, y)$ конечно порожден по определению.

Предположим, что $(x,y)$ --- главный. Тогда $\exists f(x, y): (x, y) = (f(x,y))$.

Т.к. $x \,\vdots\, f (x, y),  y \,\vdots\, f(x, y) $, то $\deg f \leq 1$.

Если $\deg f(x, y) = 0$, то $f(x, y) \in \Q$: при $f(x, y) = 0$ $(f(x, y)) = 0$, при $f(x, y) \neq 0$ $(f(x,y)) = \Q[x, y]$. Оба случая нам не подходят.

Если $\deg f(x, y) = 1$, то $\exists a, b\in \Q^*: f(x, y) = ax = by$, откуда $x = a^{-1} b y$, что тоже неверно.

б) $K = \Q[x_1, x_2, \;\ldots\; x_n, \; \ldots \;]$

$I = (x_1, x_2, \;\ldots\;, x_n, \;\ldots\;)$

Предположим, $I$ конечно-порожден, т.е $\exists f_1, \;\ldots\;, f_t \in K: I = (f_1, \;\ldots\;, f_t)$. $f_i$ можно представить в виде $x_1 g^1_{i} + \;\ldots\; + \; x_{N_i} g^{N_i}_{i}$ для некоторого $N_i \in \N$, т.к. $f_i$ лежит в идеале $(x_1,  \;\ldots\;, x_n,  \;\ldots\;)$. Положим $N = \max \{N_1, \;\ldots\;, N_t \}$, тогда $f_i = x_1 g^1_{i} + \;\ldots\; + \; x_{N} g^{N}_{i} $.

Т.к. $x_{N+1} \in I = (f_1, \;\ldots\;, f_t )$, то $\exists a_1, \;\ldots\; a_t : x_{N+1} = a_1 f_1 + \;\ldots\; a_t f_t$. Приравнивая $x_1 = \;\ldots\; = x_N = 0$ --- на них все $f_1, \;\ldots\;, f_t$ равны 0 --- и $x_{N+1} = 1$, приходим к противоречию.
 
\end{solution}

% ------------------------------------------------------------

\begin{problem}[17 (4.3)]

Умение находить факторкольца.

\end{problem}

% ------------------------------------------------------------

\begin{problem}[18 (4.8)]

Пусть $J \subset I \subset K$ — цепочка вложенных идеалов в кольце $K$. Тогда кольцо $(K/J)/(I/J)$ изоморфно $K/I$.

\end{problem}

\begin{solution}

Рассмотрим гомоморфизм $\varphi: K/J \to K/I$, $x + J \mapsto x + I$.

Гомоморфизм корректен, т.к. независимо от выбора представителя $x$ получим одно и то же: $x + J = y + J \Then x - y \in J \Then x - y \in I \Then x + I = y + I$.

$\varphi$ --- сюръекция, т.к. $\forall a + I \in K/I: \exists x = a: \varphi(a) = a + I$.

Значит, по теореме о гомоморфизме: $(K/J) / \ker \varphi \isom K/I$. 

Т.к. $\ker \varphi = \{x + J: x + I = I\}  = \{x + J: x \in I \} = I / J$, то $(K/I)/(I/J) \isom I/J$, ч.т.д..

\end{solution}

% ------------------------------------------------------------

\begin{problem}[19 (4.9)]
	
В кольце главных идеалов любой простой идеал максимален.

\end{problem}

\begin{solution}
	
Пусть $(p)$ --- простой идеал в КГИ $K$. Пусть $I$ --- идеал в $K$: $(p) \subset I \subset K$.

$I$ --- порожден одним элементом $\Then$ $I = (x)$ $\Then$ $p \,\vdots\, x$ по задаче 15а) на 3-4.

Т.к. $(p)$ --- простой идеал, то $p$ --- простой элемент (по задаче 20 на 3-4) $\Then$ $x \sim p$ или $x \in K^*$.

Если $x \sim p$ $\Then$ $I = (x) = (p)$. Если $x \in K^*$ $\Then$ $I = (x) = K$.

Таким образом, $\nexists I: (p) \subsetneq I \subsetneq K$ $\Then$ $(p)$ --- максимален.
	
\end{solution}
% ------------------------------------------------------------

\begin{problem}[20 (4.10)]
	Умение находить максимальные и простые идеалы.	
\end{problem}
\begin{solution}
	Для решения таких задач необходимы следующие теоремы:
	\begin{itemize}
		\item В факториальном кольце из неразложимости элемента следует его простота.
		\item В области целостности, $\forall p \neq 0$: (p) - простой идеал $\Leftrightarrow$ p - простой элемент
	\end{itemize}
	Пусть K - коммутативное кольцо, а I - идеал.
	\begin{itemize}
		\item K/I - область целостности $\Leftrightarrow$ I - простой идеал
		\item K/I поле $\Leftrightarrow$
		идеал I максимален $\Leftrightarrow$
		в K/I нет нетривиальных идеалов
		\item Любой максимальный идеал прост
		\item В КГИ простой идеал является максимальным.
	\end{itemize}
\end{solution}
% ------------------------------------------------------------

\begin{problem}[21 (5.3)]
a)Верно ли, что $Quot(Quot(K)) \cong Quot(K)$?

б)Пусть $K \subset L \subset Quot(K)$ верно ли, что $Quot(K) \cong Quot(L)?$ 
\end{problem}
\begin{solution}
	а) Пусть L - поле. Тогда $Quot(L) \cong L$ .
	
	Изоморфизм $\phi : L \leftrightarrow Quot(L)$ выглядит как $\phi: (a * b^{-1}) \leftrightarrow \frac{a}{b}$, где $\frac{a}{b} \in Quot(L)$.
	
	$Quot(K)$ - поле. Тогда, в частности, $Quot(Quot(K)) \cong Quot(K)$.
	
	б) $K \subset L \subset Quot(K) \Then Quot(K) \subset Quot(L) \subset Quot(Quot(K)) \cong Quot(K) \Then Quot(K) \cong Quot(L)$ 
\end{solution}
% ------------------------------------------------------------

\begin{problem}[22 (5.4)]
	Является ли кольцо $\Z[x]$ евклидовым?
\end{problem}
\begin{solution}
	В $\Z[x]$ (и в $\Z_6[x]$, например) идеал $(2x, x^2)$ не является главным. Из евклидовости кольца следует, что оно является кольцом главных идеалов. Значит, из того, что кольцо не является кольцом главных идеалов следует, что оно не является евклидовым кольцом.
\end{solution}
% ------------------------------------------------------------

\begin{problem} [23 (5.5)]
Какие многочлены степени 0:

a) неприводимы в $\Z[x]$

б) являются простыми элементами в $\Z[x]$
\end{problem}
\begin{solution}
Многочлены степени 0 в $\Z[x]$ - это $\Z \subset \Z[x]$.

а) Составные числа являются приводимыми элементами $\Z[x]$ (по определению).

Простые числа - неприводимы: $p \in \Z \subset \Z[x] \Then p = fg \Then (deg(f) \leq 0, deg(g) \leq 0) \Then deg(f) = deg(g) = 0 \Then f \in \Z, g \in \Z \Then (f \in \Z^*) \vee (g \in \Z^*) \Then$ многочлен $a$ неприводим.
 
б) Составные числа не являются простыми элементами $\Z[x]$. Рассмотрим $ab = x | x = ab$, где $|a| > 1, |b| > 1$. Ясно, что $(ab\not| a) \wedge (ab\not| b) $ в $\Z$ и в $\Z[x]$. Тогда это не простой элемент $\Z[x]$.

Простые числа являются простыми элементами $\Z[x]$: пусть $f(x)g(x) \vdots p$, но $(f(x) \not \vdots p) \wedge (g(x) \not \vdots p)$. Тогда хотя бы один коэффициент у каждого многочлена не делится на $p$. Тогда пусть $f(x) = \sum{f_k  x^k}, g(x) = \sum{g_k x^k}; f(x)g(x) = \sum{a_k x^k}; i = max(i | f_i \not\vdots p), j = max(j | g_j \not\vdots p) \Then 
a_{i+j} = (f_{i + j}g_0 + f_{i + j - 1}g_1 + ... + f_ig_j + ... + f_0g_{i + j})$ Заметим, что в этой сумме все слагаемые, кроме $f_ig_j$ делятся на $p$ т.к. по выбору $i$ и $j$, один из коэффициентов имеет номер больший максимального, не делящегося. Следовательно, $a_{i+j} \not\vdots p \Then f(x)g(x) \not\vdots p$. Получено противоречие с делимостью. 
\end{solution}
% ------------------------------------------------------------

\begin{problem} [24 (5.6)]
	
a)Примитивный многочлен $f \in \Z[x]$ ненулевой степени:
неприводим в $\Z[x]$ $\Leftrightarrow$ неприводим в $\Q[x]$

б)Произведение примитивных многочленов примитивно

в)Примитивный неприводимый многочлен в $f \in \Z[x]$ - простой элемент кольца
\end{problem}
\begin{solution}
Определим содержание $c(f)$ многочлена $f$ как НОД коэффициентов многочлена.

Таким образом, многочлен $f$ примитивен $\Leftrightarrow$ $c(f) = 1$ 

Лемма Гаусса: $\forall f, g \in \Q[x]: c(fg) = c(f) \cdot c(g)$. Она доказывается с использованием утверждения 23.

а, $\Rightarrow$) $f \in \Q[x]$ неприводим $\Rightarrow$ $f \in \Z[x] (\subset \Q[x])$ неприводим. (Если есть приведение $f=gh$ в $\Z[x]$, то оно есть и в $\Q[x]$)

а, $\Leftarrow$) пусть $f=gh$ - приведение в $\Q[x]$. $f=\frac{A}{B}\overline{g}\frac{C}{B}\overline{h}$, где $A, C$ - наибольшие общие делители коэффициентов многочленов $g$ и $h$ соответственно. $B, D$ - общий знаменатель коэффициентов. Таким образом, $\overline{g}$ и $\overline{h}$ - примитивные.
$\Then$ $f = \frac{AC}{BD}\overline{g}\overline{h}$, $BDf = AC\overline{g}\overline{h}$
$\Then$ $c(BDf) = c(BD) \cdot 1 = c(AC\overline{g}\overline{h}) = c(AC) \cdot 1 \cdot 1$
$\Then$ $AC = BD \cdot u$ и $u \in \Z^*$
$\Then$ $\overline{f} = u \overline{g}\overline{h}$ в $\Z[x]$
$\Then$ таким образом, из приводимости в $\Q[x]$, следует приводимость в $\Z[x]$, следовательно, из неприводимости в $\Z[x]$ следует неприводимость в $\Q[x]$

б) произведение примитивных примитивно
Пусть $f = c(f)f_1$, $g = c(g)g_1$. Тогда $f_1$ и $g_1$ - примитивны. 
$\Then$ $fg = c(f) \cdot c(g) \cdot f_1 \cdot g_1 = c(f)c(g)f_1g_1$. Пусть простое число p делит $fg$. Тогда по утверждению в том, что простые в $\Z$ являются простыми в $\Z[x]$ имеем $p|f \vee p|g$. Но они примитивны $\Rightarrow$ противоречие.

в) $\Q[x]$ факториально 
$\Then$ из неприводимости следует простота

$\rho(x) h(x) \vdots f(x)$ в $\Z[x]$ 
$\Then$ без ограничения общности $f(x) | \rho(x)$ в $\Q[x]$
$\Then$ $f(x) l(x) = \rho(x)$ в $\Q[x]$

$f(x) \frac{A}{B} \overline{l(x)} = \rho(x)$,
$f(x) \frac{A}{B} \overline{l(x)} = c(\rho) \overline{\rho(x)}$
$\Then$ $Af(x)\overline{l(x)} = B c(\rho) \overline{\rho(x)}$
$\Then$ $A \~ B c(\rho)$ по лемме Гаусса
$\Then$ $A \cdot u = B \cdot c(\rho), u \in \Z[x]^*$
$\Then$ $f(x)\overline{l(x)} = u \overline{\rho(x)}$
$\Then$ $f(x) | \overline{\rho(x)}$ в $\Z[x]$, т.е. получена простота в $\Z[x]$ из простоты в $\Q[x]$
  
\end{solution}
% ------------------------------------------------------------

\begin{problem}[25 [Каргальцев]]
Докажите, что в кольце главных идеалов любая возрастающая цепочка идеалов

$$ (a_1) \subset (a_2) \subset \ldots \subset (a_n) \subset \ldots $$

стабилизируется, то есть найдется такое $k$, то $(a_k) = (a_{k + 1}) = \ldots$
\end{problem}

\begin{solution}
Поскольку \((a_i) \subset (a_{i + 1}) \Then a_{i + 1} | a_i\).

Возьмем \(I = \cup_{k = 1}^{\infty} (a_k)\). покажем, что \(I\) -- идеал. Пусть \(a \in I, b \in I \Then \exists k_1, k_2: a \in (a_{k_1}), b \in (a_{k_2})\). Тогда положим \(k = max(k_1, k_2)\). \(a, b \in (a_k) \Then (a + b) \in (a_k) ((a_k) \textup{--- идеал}) \Then (a + b) \in I\). Аналогично \(\forall x \in K xa \in (a_k) \Then xa \in I\).

Поскольку \(K\) --- КГИ, то существует \(x: I = (x)\). \(x \in I \Then \exists k: x \in (a_k)\). Но \(a_k \in (x)\). Тогда \(x | a_k \land a_k | x \Then x \sim a_k\). Но в силу вложенности это верно и для всех \(j > k\), то есть \(\forall j \geqslant k a_j \sim a_k \Then (a_j) = (a_k)\). То есть цепочка действительно стабилизируется.
\end{solution}

% ------------------------------------------------------------


\begin{problem} [26 (4.8)]
	В КГИ из простоты идеала следует его максимальность
\end{problem}
\begin{solution}
	Простота идеала: $\forall ab \in I \rightarrow ((a \in I) \vee (b \in I))$
	
	От противного: пусть $I \subset J \subset K$, где $I = (p)$ - простой идеал над кольцом $K$. Тогда $p \in I \subset J = (x) \Then p \in J$. КГИ является факториальным кольцом (вопрос 6 на 7-8), следовательно: $(p)$ - простой идеал $\Leftrightarrow$ p - простой элемент (задача 20 на 3-4). Тогда $x \~ p$ или $x \in K^*$ (из $p=kx$). Тогда:
	\begin{itemize}
		\item $x \~ p \Then I = J$
		\item $x \in K^* \Then J = (x) = (1) = K$ 
	\end{itemize} 
	 
\end{solution}
% ------------------------------------------------------------

\begin{problem} [27]
	Любой максимальный идеал прост
\end{problem}
\begin{solution}
	Теорема 1 (вопрос 21 (4.6) на 3-4): K/I - поле $\Leftrightarrow$ нетривиальный идеал I - максимален $\Leftrightarrow$ K/I не имеет нетривиальных идеалов
	
	Теорема 2 (вопрос 22 (4.7) на 3-4): K/I - область целостности $\Leftrightarrow$ I - прост
	
	Поле является областью целостности ($ab=0 \Then a = 0*b^{-1}= 0$ или $b = 0$). Тогда получим: I - максимален $\Rightarrow$ K/I - поле $\Rightarrow$ I - прост
\end{solution}
% ------------------------------------------------------------

\begin{problem} [28]
	Признак неприводимости Эйзенштейна для простого идеала $I$ и факториального кольца $K$
\end{problem}
\begin{solution}
	\begin{itemize}
	\item Формулировка: Пусть $K$ - факториальное кольцо, $I \subset K$ - простой идеал, $f(x) = \sum\limits_{k=0}^{n} a_kx^k  \in K[x]$ - многочлен степени $n \geq 1$. Пусть $a_0...a_{n-1} \in I$, $a_0 \not\in I^2, a_n \not\in I$. Тогда у $f(x)$ нет делителей степени $d$ при $1 \leq d \leq n-1$
	
	\item Доказательство (c консультации) повторяет доказательство признака Эйзенштейна для многочленов с целыми коэффициентами: пусть $f(x) = g(x)h(x), 0 < deg(g(x)) < n, 0 < deg(h(x)) < n, g(x) \in K[x], h(x) \in K[x]$.
	
	$g(x) = \sum\limits_{k=0}^{deg(g(x))} g_kx^k$, $h(x) = \sum\limits_{k=0}^{deg(h(x))} h_kx^k$. По условию $a_0 = g_0h_0 \not\in I^2 \Then (g_0 \not\in I) \vee (h_0 \not\in I)$. Без ограничения общности $g_0 \not\in I$ и $h_0 \in I$. Пусть $h_0...h_k \in I, h_{k+1} \not\in I$. Рассмотрим коэффициент при $x^{k+1}$: $a_{k+1} = g_0h_{k+1} + g_1h_k + ... + g_{k+1}h_0$. По предположению все слагаемые, кроме первого, лежат в $I$, а $g_0h_{k+1}$ не лежит. Но тогда $a_{k+1} \not\in I \Then k+1 = n \Then deg(h(x)) \geq k + 1 = n$. Получено противоречие со степенью многочлена $h(x)$.
	\end{itemize}
\end{solution}
% ------------------------------------------------------------

\begin{problem} [?? (?.?)]
\end{problem}
\begin{solution}
\end{solution}
% ------------------------------------------------------------

\begin{problem}[?? [Каргальцев]]
\end{problem}

\begin{solution}
	а) Если \(z\) --- неразложимый элемент \(D\), то существует такое простое целое число \(p\), что \(N(z) = p\) или \(N(z) = p^2\)
	
	\(N(z) = z\bar{z}\). Разложим \(N(z)\) в произведение простых как натуральное число:
	
	\(z\bar{z} = N(z) = p_1^{\alpha_1} \cdot \ldots \cdot p_n^{\alpha_n}\).
	
	Так как \(z\) неразложим, а \(D\) --- евклидово, то \(z\) --- прост, значит \(\exists k: z | p_k\).
	
	\(p_k = zu \Then p_k = \bar{p_k} = \bar{z}\bar{u} \Then \bar{z} | p_k \Then N(z) | p_k^2 \Then N(z) = 1, p_k\) или \(p_k^2\). Но так как если \(N(z) = 1\), то \(z\) --- обратим (а, следовательно, неразложим), то \((z) = p_k \lor N(z) = p_k^2\).
	
	б) Если \(z\) --- неразложимый элемент \(D\) и \(N(z) = p^2\), то \(z \sim p\).
	
	Пусть \(\bar{z} = ab \Then z = \bar{a} \bar{b} \Then \bar{z}\) --- неразложим.
	
	\(z \bar{z} = N(z) = p \cdot p\). В силу единственности разложения на неразложимые, \(z \sim p\).
	
	в) Если \(N(z) = p\), то \(z\) --- неразложимый элемент \(D\).
	
	в \(D a|b \Then N(a) | N(b)\).
	
	Пусть \(a | z \Then N(a) | N(z)\). В силу простоты \(N(z)\) либо \(N(a) = 1\) и, следовательно, \(a\) --- обратимый, либо \(N(a) = N(z)\) и тогда \(a \sim z\). То есть \(z\) неразложим.
	
	г) Пусть \(p\) --- простое целое число. Тогда есть два варианта: либо \(p\) неразложимо в \(D\), либо \(p\) = \(z\bar{z}\), где \(z\) -- неразложимо в \(D\). Таким образом описываются все неразложимые элементы \(D\).
	
	Пусть \(p\) разложимо в \(D\). Тогда найдется такой неразложимый \(z: z|p\). Поскольку \(z\) не ассоциирован с \(p\), \(N(z) \ne N(p) \Then N(z) = p\). Тогда \(z\) -- неразложимый и \(z\bar{z} = N(z) = p\).
	
	Любой неразложимый элемент \(D\) --- либо простое целое число, либо его норма --- простое целое число.
\end{solution}

\end{document}
